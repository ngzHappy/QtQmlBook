%使用xelatex编译
%版权所有,翻版必究
%本文件由程序自动生成,任何修改将被覆盖






\cleardoublepage                              %增加空白页
\setcounter{secnumdepth}{-2}                  %暂停编号,但加入目录
\chapter{
前言
}
\setcounter{secnumdepth}{3}                   %恢复编号


%简单介绍Qt...
Qt往往被认为是一套跨平台的图形界面开发架构。
诚然,Qt对于图形界面支持的很好,并且,这一方面被越来越多的团队所接纳。
但Qt并不仅限于开发图形界面,它其实是一种更加通用的客户端开发架构。
Qt几乎提供了用于构建一个客户端所需的所有模块,包括但不限于
蓝牙模块、
串口模块、
音频模块、
网络模块、
多媒体模块、
数据库模块……

更加令用户愉悦的是,由于Qt本身是被广泛使用的开源产品,
用户可以轻松的享受到来自整个开源社区(其中包括整个C/C++社区)的加持。
也就是,
即使Qt本身并未提供一些方面的支持(或者Qt自身提供的支持无法满足要求),
用户也可以轻松的找到免费或付费的解决方案。
即使有些情况下用户无法找到解决麻烦的现成并有效的手段,
但至少通过社区,
用户可以获得一些走出困境的灵感。

%简单介绍QML+QtQuick...
随着新的硬件设备的广泛采用和开发者观念的变更。
完全采用C++开发图形界面变得越来越笨手笨脚,
并且最终效果亦不佳。
所幸的是,Qt一直没有停下前行的脚步。
Qml以及基于Qml的Qt Quick被引入和大力推广。
用户可以安心的用C++做基础模块,
而利用Qt Quick将一切快速的组织起来。

Qt Quick比传统的Qt Widgets不仅仅更加有效利用CPU多核资源(Qt Quick可以异步渲染)。
更令人高兴的是,Qt Quick完全是在显卡端完成渲染。
即使某些设备不支持显卡渲染,Qt自身也可以通过软件模拟达到效果。
这一切并不受限于某几个平台,而是几乎所有平台。
智能手机、个人电脑、嵌入式设备,它们都受到支持。
用户可以使用Qt Quick敏捷的构造出美观、高效、稳健并跨平台的一流产品。

Qt公司为用户提供了大量的辅助工具。
用这些工具,用户可以迅速的编写、测试、调试、部署、以及调优和美化。
除了Qt公司直接提供的工具外,
由于Qt的广泛使用,
很多第三方工具链也支持Qt。
虽然,到目前为止,
这些第三方支持主要是面向传统Qt C++。
但仅仅来自Qt自身的工具链对于Qt Quick的支持也不会令用户失望。


%简单介绍如何阅读本书...
本书是一本完整介绍Qt Quick的书。
通过本书,读者可以完整的掌握整个Qt Quick的全貌。
但限于篇幅和个人精力所限,一些细节可能被舍弃。

读者在阅读本书之前应当对于Qt C++、JavaScript和OpenGL具有一定了解,
并具备一定的图形学相关知识。
为了避免本书变成数千页的大部头,
本书并不会对上述细节多做解释。

基于Qt Qml的另一个模块是Qt 3D。
Qt 3D和Qt Quick是两个几乎不关联的模块,
虽然它们都基于Qt Qml。
本书并不介绍Qt 3D。

本书
第一章带领读者纵览整个Qt Quick,
对于Qt Quick不太熟悉的读者可能读起来有些吃力。
对于第一章,阅读起来有些困难的地方直接跳过即可。

本书
第二章介绍Qt Qml基础语法以及Qt Quick基本元素,
第三章介绍如何使用C++扩展Qt Quick,
第九章介绍Qt Quick基础控件,
这三章是主干章节。

第四章介绍Qt Quick动画和状态机,
第五章介绍Qt Quick粒子系统,
第六章介绍一些常见特效,
第八章介绍Qt Quick的图文表模块,
第十章介绍Qt Quick的模型视图模块。
第七章本书介绍如何结合FFMPEG构建多媒体模块。
这些章节各有主题,读者根据需要选读即可。










%使用xelatex编译
%版权所有,翻版必究
%本文件由程序自动生成,任何修改将被覆盖



