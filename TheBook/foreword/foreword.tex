%使用xelatex编译
%版权所有,翻版必究
%本文件由程序自动生成,任何修改将被覆盖






\cleardoublepage                              %增加空白页
\setcounter{secnumdepth}{-2}                  %暂停编号,但加入目录
\chapter{
前言
}\label{c000020}
\setcounter{secnumdepth}{4}                   %恢复编号,目录深度为4


%简单介绍Qt...
Qt往往被认为是一套跨平台的图形界面开发架构。
诚然,Qt对于图形界面支持的很好,并且,这一方面被越来越多的团队所接纳。
但Qt并不仅限于开发图形界面,它其实是一种更加通用的客户端开发架构。
Qt几乎提供了用于构建一个客户端所需的所有模块,包括但不限于
蓝牙模块、
串口模块、
音频模块、
网络模块、
多媒体模块、
数据库模块……

更加令用户愉悦的是,由于Qt本身是被广泛使用的开源产品,
用户可以轻松的享受到来自整个开源社区(其中包括整个C{\sourcefontone{}/}C{\sourcefontone{}+}{\sourcefontone{}+}社区)的加持。
也就是,
即使Qt本身并未提供一些方面的支持(或者Qt自身提供的支持无法满足要求),
用户也可以轻松的找到免费或付费的解决方案。
即使有些情况下用户无法找到解决麻烦的现成并有效的手段,
但至少通过社区,
用户可以获得一些走出困境的灵感。

%简单介绍QML+QtQuick...
随着新的硬件设备的广泛采用和开发者观念的变更。
完全采用C{\sourcefontone{}+}{\sourcefontone{}+}这类静态计算机语言开发图形界面变得越来越笨手笨脚,
并且最终效果亦不佳,
很多由动态语言轻松可以达到的效果往往用静态计算机语言难以实现。
所幸的是,Qt一直没有停下前行的脚步。
Qml以及基于Qml的Qt Quick被引入和大力推广。
Qml本身被设计为一种简单而优雅的脚本语言,并且,
Qml天然支持一个JavaScript子集
\footnote{
主要是在Qml中禁用了JavaScript中this这个
语义模糊的关键字,并禁用JavaScript中的全局变量。
}。
用户可以安心的用C{\sourcefontone{}+}{\sourcefontone{}+}做基础模块,
而利用Qt Quick将一切快速的组织起来。


Qt Quick比传统的Qt Widgets不仅仅更加有效利用CPU多核资源
(Qt Quick可以异步渲染)。
更令人高兴的是,Qt Quick完全是在显卡端完成渲染。
即使某些设备不支持显卡渲染,Qt自身也可以通过软件模拟达到效果。
这一切并不受限于某几个平台,而是几乎所有平台。
智能手机、个人电脑、嵌入式设备,它们都受到支持。
用户可以使用Qt Quick敏捷的构造出美观、高效、稳健并跨平台的一流产品。

Qt公司为用户提供了大量的辅助工具。
用这些工具,用户可以迅速的编写、测试、调试、部署、以及调优和美化。
除了Qt公司直接提供的工具外,
由于Qt的广泛使用,
很多第三方工具链也支持Qt。
虽然,到目前为止,
这些第三方支持主要是面向传统Qt C{\sourcefontone{}+}{\sourcefontone{}+}。
但仅仅来自Qt自身的工具链对于Qt Quick的支持也不会令用户失望。


%简单介绍如何阅读本书...
本书是一本完整介绍Qt Quick的书。
通过本书,读者可以完整的掌握整个Qt Quick的全貌。
但限于篇幅和个人精力所限,一些细节可能被舍弃。

读者在阅读本书之前应当对于Qt C{\sourcefontone{}+}{\sourcefontone{}+}、JavaScript和OpenGL具有一定了解,
并具备一定的图形学相关知识。
为了避免本书变成数千页的大部头,
本书并不会对上述细节多做解释。

基于Qt Qml的另一个模块是Qt 3D。
Qt 3D和Qt Quick是两个几乎不关联的模块,
虽然它们都基于Qt Qml。
本书并不介绍Qt 3D。

本书采用C{\sourcefontone{}+}{\sourcefontone{}+}17标准编写,
Qt最低版本为Qt 5.12.0,
FFMPEG版本最低为4.1,
Boost版本为1.69.0,
如果读者在编译本书代码出现了问题,
请尝试查询当前开发环境是否正确。
本书只支持桌面Windows和桌面Linux,
这两个平台已经足够涵盖绝大多数读者,并能够完整诠释所有技术细节。
对于刚刚接触Qt Quick的读者,
一方面太多平台细节会成为干扰读者统揽全局的噪音;
另一方面对于一个具体的平台,
本书采用的一些技术特性可能不被支持
\footnote{
一些平台可能只支持有限的C{\sourcefontone{}+}{\sourcefontone{}+}17标准和OpenGL特性,
另一方面在一些平台下FFMPEG也难以编译,
而且Qt自带的多媒体库在特定平台下如何部署也千差万别,
甚至有些平台可能根本不支持Qt 5.12.0及其后续版本而只支持一些老的Qt版本。
},
这些细节将极大拖慢本书的写作进度和涵盖的范围。
为了向读者展示一个全面的并且现代的Qt,本书不得不舍弃过多的平台适配而轻装上阵。

本书
第 \ref{c000010}章带领读者纵览整个Qt Quick,
对于Qt Quick不太熟悉的读者可能读起来有些吃力。
对于第 \ref{c000010}章,
初次阅读起来有些困难的地方直接跳过即可。

本书
第 \ref{c000011}章介绍Qt Qml基础语法以及Qt Quick基本元素,
第 \ref{c000012}章介绍如何使用C{\sourcefontone{}+}{\sourcefontone{}+}扩展Qt Quick,
第 \ref{c000018}章介绍Qt Quick基础控件,
这三章是主干章节。

第 \ref{c000013}章介绍Qt Quick动画和状态机,
第 \ref{c000014}章介绍Qt Quick粒子系统,
第 \ref{c000015}章介绍一些常见特效,
第 \ref{c000017}章介绍Qt Quick的图文表模块,
第 \ref{c000019}章介绍Qt Quick的模型视图模块。
第 \ref{c000016}章本书介绍如何结合FFMPEG构建多媒体模块。
这些章节各有主题,读者根据需要选读即可。



















%使用xelatex编译
%版权所有,翻版必究
%本文件由程序自动生成,任何修改将被覆盖



