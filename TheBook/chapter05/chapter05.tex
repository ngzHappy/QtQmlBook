
%使用XeLaTeX编译
%版权所有,翻版必究
%本文件由程序自动生成,任何修改将被覆盖
%2019 年 01 月 23 日




%\FloatBarrier
\cleardoublepage
\chapter{
粒子系统
}\label{c000014}


Qt Quick自带一个艺术级的粒子系统模块。
读者可以用它实现一些绚丽的效果,比如:
模拟烟雾、
模拟流星、
模拟火焰……

不过读者如果需要工业级
或者类似的要求十分严苛的
粒子仿真系统,
Qt Quick自带的粒子系统显然
不是一个好的选择。

本书不打算介绍Qt Quick自定义
粒子系统。
因为,对于常见
粒子效果,Qt Quick可以轻松实现。
而对于那些不得不使用Qt Quick自定义
粒子系统实现的复杂效果,没有必要
非要在Qt Quick粒子系统里
绕圈,直接使用3D API实现即可。

%导引
\FloatBarrier
\section{
导引
}\label{c000014s01}


\begin{comment}
https://www.kancloud.cn/cloudcastle/qt5-demo/109870
Using the Qt Quick Particle System
http://doc.qt.io/qt-5/qtquick-effects-particles.html
\end{comment}

要使用Qt Quick Particles体系只需要在Qml文件
开头添加\begin{littlelongworld}
import QtQuick.Particles 2.12
\end{littlelongworld}
即可。

Qt Quick粒子系统由
粒子绘制器(ParticlePainter)、
粒子组(ParticleGroup)、
系统(ParticleSystem)、
发射器(Emitter)
以及
影响器(Affector)
构成。

逻辑粒子(Logical Particle)
在一个时刻只被分到一个粒子组。
在运行时,逻辑粒子的粒子组可以改变。

在一个时刻,一个粒子组只对应于一个
粒子绘制器。

因而,在运行时,改变逻辑粒子的粒子组可以
改变逻辑粒子的外观但不改变逻辑粒子的运动
状态。

特别的,如果粒子组没有对应粒子绘制器,
那么属于该粒子组的逻辑粒子不可见。

发射器可以给逻辑粒子一个初始位置、寿命、速度以及加速度,
并影响逻辑粒子的密度。

影响器可以在粒子发射之后改变
粒子的运行状态。

系统可以可以从全局监视和控制粒子系统的运行状态。


% ______all_key_words
% the_book_chapter the_book_subsection the_book_subsubsection
% the_book_section the_book_image the_book_table
% the_book_file the_book_tree_file the_book_command_file
% littlelongworld tabbing ref
% figurename tablename filesourcenumbernameone
% treeindexnumbernameone commandnumbernameone footnote
% item itemize comment textbullet
% \hspace*{\parindent}







%使用XeLaTeX编译
%版权所有,翻版必究
%本文件由程序自动生成,任何修改将被覆盖
%2019 年 01 月 23 日



