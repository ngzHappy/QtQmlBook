
%用于快速测试

\documentclass[12pt,hyperref,UTF8]{ctexbook}

\let\counterwithout\relax
\let\counterwithin\relax

%@P143
\usepackage{xcolor}        %导入包xcolor
\usepackage[
a4paper ,
left=3.0cm,                %靠近装订线的边距
right=3.0cm,               %远离装订线的边距
top=2.0cm,
bottom=2.0cm,
headheight=1.3cm,
headsep=0.5cm,
marginparsep=0.8cm,
marginparwidth=2.2cm
]{geometry}                %导入包geometry

 %导入常用包
\usepackage{graphicx}
\usepackage{float}
\usepackage{amsmath}
\usepackage{cite}
\usepackage{caption}
\usepackage{titlesec}
\usepackage{chngcntr}
\usepackage{setspace}
\usepackage{tocbibind}  %设置目录
\usepackage{tocloft}
\usepackage{multicol}

\usepackage{amsthm}
\usepackage{xltxtra}
\usepackage{listings}

\usepackage{fontspec}
%\usepackage{ebgaramond}

\setmainfont{Times New Roman}
\setsansfont{DejaVu Sans}
\setmonofont{Latin Modern Mono}

%\usepackage{newtxtext, newtxmath}
%\setsansfont{Times New Roman}
%\setmonofont{Times New Roman}
%\usepackage{txfonts}
\definecolor{gray1}{rgb}{0.9,0.9,0.85}
\newfontfamily\unicodefont{Consolas}
%\setCJKmainfont{Times New Roman}


%%常见符号
\usepackage{wasysym}    %
\usepackage{textcomp}   %
\usepackage{pifont}     %

\usepackage{color}
\definecolor{colorbackgroundthisproject}{rgb}{1,1,1} %页面背景颜色
\definecolor{colortextthisproject}{rgb}{0,0,0}       %文字颜色
%设置页面颜色
\pagecolor{colorbackgroundthisproject}
%设置字体颜色
\color{colortextthisproject}

\usepackage{xhfill}
%\usepackage{times}

\usepackage{marginnote}
\usepackage{varwidth}

%设置输出pdf格式
\usepackage[
    colorlinks=true ,
    %bookmarks=true,
    %bookmarksopen=false,
    %pdfpagemode=FullScreen,
    %pdfstartview=Fit,
    bookmarksnumbered=true,
    pdftitle={Qml} ,       %标题
    pdfauthor={Qml} ,      %作者
    pdfsubject={Qml} ,     %主题
    pdfkeywords={Qml} ,    %关键字
    linkcolor=colortextthisproject
]{hyperref}


\usepackage{longtable}
\usepackage{booktabs}

\usepackage{listings}%引入程序代码

%http://blog.sina.com.cn/s/blog_5e16f1770100kv5t.html
\lstset{
language=C,
breaklines=true,
basicstyle=\small\unicodefont,
stringstyle=\small\unicodefont ,
keywordstyle=\small\unicodefont ,
commentstyle=\small\unicodefont ,
identifierstyle=\small\unicodefont ,
numbers=left, 
numberstyle=\footnotesize,
frame=single,
backgroundcolor=\color{gray1},
showstringspaces=true
} 

 
 

\title{这是一本书}                              %书籍名称
\author{Good Luck}                              %作者名

%设置ctex
%P135
\CTEXsetup[ number={ \arabic{chapter} } ]{chapter}
%\CTEXsetup[ number={ \arabic{section} } , name={第,节} ]{section}
 
\RequirePackage{fancyhdr}
\RequirePackage{zhnumber}
                
\RequirePackage{titlesec, titletoc}
\RequirePackage{tikz,pgf}
\usetikzlibrary{shapes,calc}
% 参考:http://www.latexstudio.net/archives/8967.html
% 拇指
 

 %解决BUG
\makeatletter
\renewcommand{\numberline}[1]{%
\settowidth\@tempdimb{#1\hspace{0.5em}}%
\ifdim\@tempdima<\@tempdimb%
  \@tempdima=\@tempdimb%
\fi%
\hb@xt@\@tempdima{\@cftbsnum #1\@cftasnum\hfil}\@cftasnumb}
\makeatother

\begin{document}


\frontmatter
%\pagestyle{empty}                 %关闭页眉页脚
\maketitle                        %生成封面

\cleardoublepage

\pagestyle{headings}              %开启页眉页脚
\pagenumbering{roman}             %重新开始页码编号

\tableofcontents                  %生成目录
 
 

\mainmatter

 
 

\cleardoublepage
%
%    arabic - 阿拉伯数字
%    roman - 小写的罗马数字
%    Roman - 大写的罗马数字
%    alph - 小写的字符形式
%    Alph -大写的字符形式
%
\pagenumbering{arabic}  %重新开始页码编号
\chapter{ Qchapter 1}
%P129
\section{蜘蛛超人}
\label{testsection}

\subsection{subsection 1}

{\small\unicodefont{sudo mkfontscale}\\}
{\small\unicodefont{sudo mkfontdir}\\}
{\small\unicodefont{fc-cache -v}}

QQQQQQQQQQQQQ
\textit{在文字录入}\footnote{jjjjjx 141656}比赛(打字比赛)中,\textsl{最公平的比赛用}文本就是随机文本,这个随机汉字生成器便是为此所作。普通人的汉字录入速度一般是每分钟几十个到一百多个,我们可以生成一两千字的随机汉字文本,让比赛者录入完这些汉字,依据他们的比赛用时和正确率判断名次。生成随机汉字的原始文字一般选择常用汉字,经过随机排列之后只能一个字一个字的输入,对参赛者来说是相对公平的方案。

%\begin{table}[htb]
%\marginnote{sdfds }
%\centering 

\renewcommand\arraystretch{0.7}
\setlength\belowrulesep{0pt}
\setlength\aboverulesep{0pt}

\begin{longtable}{lcr}

%表头....
\toprule 
A & B & C \marginnote{sdfds }
\\ \midrule 
\endfirsthead

%表尾...
  
\endlastfoot



%重复表头 \renewcommand\arraystretch{0.0}
\multicolumn{1}{@{}l@{}}{\tiny\noindent{}续接上表 } \\
\toprule
A & B & C 
\\ \midrule
\endhead

%重复表尾
\midrule
\endfoot 
 
1 & 2 & 3 \\
4 & 5 & 6 \\
1 & 2 & 3 \\
4 & 5 & 6 \\
1 & 2 & 3 \\
1 & 2 & 3 \\
4 & 5 & 6 \\
1 & 2 & 3 \\
4 & 5 & 6 \\
1 & 2 & 3 \\
1 & 2 & 3 \\
4 & 5 & 6 \\
1 & 2 & 3 \\
4 & 5 & 6 \\
1 & 2 & 3 \\
1 & 2 & 3 \\
4 & 5 & 6 \\
1 & 2 & 3 \\
4 & 5 & 6 \\
1 & 2 & 3 \\
1 & 2 & 3 \\
4 & 5 & 6 \\
1 & 2 & 3 \\
4 & 5 & 6 \\
1 & 2 & 3 \\
1 & 2 & 3 \\
4 & 5 & 6 \\
1 & 2 & 3 \\
4 & 5 & 6 \\
1 & 2 & 3 \\
1 & 2 & 3 \\
4 & 5 & 6 \\
1 & 2 & 3 \\
4 & 5 & 6 \\
1 & 2 & 3 \\
1 & 2 & 3 \\
4 & 5 & 6 \\
1 & 2 & 3 \\
4 & 5 & 6 \\
1 & 2 & 3 \\
1 & 2 & 3 \\
4 & 5 & 6 \\
1 & 2 & 3 \\
4 & 5 & 6 \\
1 & 2 & 3 \\
1 & 2 & 3 \\
4 & 5 & 6 \\
1 & 2 & 3 \\
4 & 5 & 6 \\
1 & 2 & 3 \\
1 & 2 & 3 \\
4 & 5 & 6 \\
1 & 2 & 3 \\
4 & 5 & 6 \\
1 & 2 & 3 \\

\bottomrule 
\caption{xxxxxx1}\label{r234} 

\end{longtable}
%\caption[xxxxxxx]{xxxxxx1 }
%\end{table}


{
test \itshape test \ttfamily test \upshape test
}

%程序 \ref{xfwaef}  
\begin{lstlisting}[
label=xfwaef,
caption=dsgdsg ,
firstnumber=100  ] 
#include <iostream> 
using namespace std;
 int main() { 
     cout<<"h ello"<<endl; //今天
     foo<int>();
     return 0;
} 
\end{lstlisting}
\renewcommand\thelstnumber{\ifnum\value{lstnumber}>3{\ }\else{\arabic{lstnumber}}\fi}
%程序 \ref{xfwaef}  
\begin{lstlisting}[
label=xfwaef,
caption=dsgdsg ,
firstnumber=1  ] 
#include <iostream> 
using namespace std;
 int main() { 
     cout<<"h ello"<<endl; //今天
     foo<int>();
     return 0;
} 
\end{lstlisting}
\renewcommand\thelstnumber{\arabic{lstnumber}}

%程序 \ref{xfwaef}  
\begin{lstlisting}[
label=xfwaef,
caption=dsgdsg ,
firstnumber=100 ] 
#include <iostream> 
using namespace std;
 int main() { 
     cout<<"h ello"<<endl; //今天
     foo<int>();
     return 0;
} 
\end{lstlisting}

%\marginnote{\fbox{程序\ \ref{x1fwaef}}}
%程序 \ref{x1fwaef}  

\begin{lstlisting}[
escapeinside={(***@}{@***)},
label=x1fwaef,
caption=sdglj,
title=程序清单 ,
nolol=false,
numbers=none ] 
#include <iostream> 
using namespace std;
 int main() { 
     cout<<"h ello"<<endl; //今天
     foo<int>();
     return 0;
} (***@\marginpar[\hfill\fbox{\begin{varwidth}{10cm}\setlength{\baselineskip}{5pt}\hfill{}fsdlf\\dfasfds\end{varwidth}}]{aabbcc}@***)
\end{lstlisting} 


{
\setlength\fboxsep{2pt}
\fbox{
    \footnotesize{\kaishu\parbox{1em}{\setlength{\baselineskip}{2pt}楷书}}\footnotesize{1.2.3}

}


dsfsdfdsa 

dsfsdfdsa 

dsfsdfdsa 

\fbox{\begin{varwidth}{10cm}\setlength{\baselineskip}{5pt}fsdlf\\dfasfds\end{varwidth}}

sdfdsafdsaf

\ref{x1fwaef}

在文字录入A\ref{testsection}A\pageref{testsection}A比赛(打字比赛)中,最公平的比赛用文本就是随机文本,这个随机汉字生成器便是为此所作。普通人的汉字录入速度一般是每分钟几十个到一百多个,我们可以生成一两千字的随机汉字文本,让比赛者录入完这些汉字,依据他们的比赛用时和正确率判断名次。生成随机汉字的原始文字一般选择常用汉字,经过随机排列之后只能一个字一个字的输入,对参赛者来说是相对公平的方案。

\begin{figure}[ht]\marginnote{\fbox{图片\ 1.22}}
\centering 
\includegraphics[width=\textwidth]{0000.jpg}
\caption[xxxxxxx]{xxxxxx }
\label{xxxxxx1}

\end{figure}%\protect\marginpar{\fbox{图片\ 1.22}}

\subsection{subsection 2}

在文字B\ref{xxxxxx1}B\pageref{xxxxxx1}B录入比赛(打字比赛)中,最公平的比赛用文本就是随机文本,这个随机汉字生成器便是为此所作。普通人的汉字录入速度一般是每分钟几十个到一百多个,我们可以生成一两千字的随机汉字文本,让比赛者录入完这些汉字,依据他们的比赛用时和正确率判断名次。生成随机汉字的原始文字一般选择常用汉字,经过随机排列之后只能一个字一个字的输入,对参赛者来说是相对公平的方案。

在文字录入比赛(打字比赛)中,最公平的比赛用文本就是随机文本,这个随机汉字生成器便是为此所作。普通人的汉字录入速度一般是每分钟几十个到一百多个,我们可以生成一两千字的随机汉字文本,让比赛者录入完这些汉字,依据他们的比赛用时和正确率判断名次。生成随机汉字的原始文字一般选择常用汉字,经过随机排列之后只能一个字一个字的输入,对参赛者来说是相对公平的方案。

\section{section 2}

\subsection{subsection 1}

在文字录入比赛(打字比赛)中,最公平的比赛用文本就是随机文本,这个随机汉字生成器便是为此所作。普通人的汉字录入速度一般是每分钟几十个到一百多个,我们可以生成一两千字的随机汉字文本,让比赛者录入完这些汉字,依据他们的比赛用时和正确率判断名次。生成随机汉字的原始文字一般选择常用汉字,经过随机排列之后只能一个字一个字的输入,对参赛者来说是相对公平的方案。

在文字录入比赛(打字比赛)中,最公平的比赛用文本就是随机文本,这个随机汉字生成器便是为此所作。普通人的汉字录入速度一般是每分钟几十个到一百多个,我们可以生成一两千字的随机汉字文本,让比赛者录入完这些汉字,依据他们的比赛用时和正确率判断名次。生成随机汉字的原始文字一般选择常用汉字,经过随机排列之后只能一个字一个字的输入,对参赛者来说是相对公平的方案。

在文字录入比赛(打字比赛)中,最公平的比赛用文本就是随机文本,这个随机汉字生成器便是为此所作。普通人的汉字录入速度一般是每分钟几十个到一百多个,我们可以生成一两千字的随机汉字文本,让比赛者录入完这些汉字,依据他们的比赛用时和正确率判断名次。生成随机汉字的原始文字一般选择常用汉字,经过随机排列之后只能一个字一个字的输入,对参赛者来说是相对公平的方案。

\subsection{subsection 2}

在文字录入比赛(打字比赛)中,最公平的比赛用文本就是随机文本,这个随机汉字生成器便是为此所作。普通人的汉字录入速度一般是每分钟几十个到一百多个,我们可以生成一两千字的随机汉字文本,让比赛者录入完这些汉字,依据他们的比赛用时和正确率判断名次。生成随机汉字的原始文字一般选择常用汉字,经过随机排列之后只能一个字一个字的输入,对参赛者来说是相对公平的方案。

在文字录入比赛(打字比赛)中,最公平的比赛用文本就是随机文本,这个随机汉字生成器便是为此所作。普通人的汉字录入速度一般是每分钟几十个到一百多个,我们可以生成一两千字的随机汉字文本,让比赛者录入完这些汉字,依据他们的比赛用时和正确率判断名次。生成随机汉字的原始文字一般选择常用汉字,经过随机排列之后只能一个字一个字的输入,对参赛者来说是相对公平的方案。


\cleardoublepage
\chapter{chapter 2}

\section{section 1}

\subsection{subsection 1}

在文字录入比赛(打字比赛)中,最公平的比赛用文本就是随机文本,这个随机汉字生成器便是为此所作。普通人的汉字录入速度一般是每分钟几十个到一百多个,我们可以生成一两千字的随机汉字文本,让比赛者录入完这些汉字,依据他们的比赛用时和正确率判断名次。生成随机汉字的原始文字一般选择常用汉字,经过随机排列之后只能一个字一个字的输入,对参赛者来说是相对公平的方案。

在文字录入比赛(打字比赛)中,最公平的比赛用文本就是随机文本,这个随机汉字生成器便是为此所作。普通人的汉字录入速度一般是每分钟几十个到一百多个,我们可以生成一两千字的随机汉字文本,让比赛者录入完这些汉字,依据他们的比赛用时和正确率判断名次。生成随机汉字的原始文字一般选择常用汉字,经过随机排列之后只能一个字一个字的输入,对参赛者来说是相对公平的方案。

在文字录入比赛(打字比赛)中,最公平的比赛用文本就是随机文本,这个随机汉字生成器便是为此所作。普通人的汉字录入速度一般是每分钟几十个到一百多个,我们可以生成一两千字的随机汉字文本,让比赛者录入完这些汉字,依据他们的比赛用时和正确率判断名次。生成随机汉字的原始文字一般选择常用汉字,经过随机排列之后只能一个字一个字的输入,对参赛者来说是相对公平的方案。

在文字录入比赛(打字比赛)中,最公平的比赛用文本就是随机文本,这个随机汉字生成器便是为此所作。普通人的汉字录入速度一般是每分钟几十个到一百多个,我们可以生成一两千字的随机汉字文本,让比赛者录入完这些汉字,依据他们的比赛用时和正确率判断名次。生成随机汉字的原始文字一般选择常用汉字,经过随机排列之后只能一个字一个字的输入,对参赛者来说是相对公平的方案。

在文字录入比赛(打字比赛)中,最公平的比赛用文本就是随机文本,这个随机汉字生成器便是为此所作。普通人的汉字录入速度一般是每分钟几十个到一百多个,我们可以生成一两千字的随机汉字文本,让比赛者录入完这些汉字,依据他们的比赛用时和正确率判断名次。生成随机汉字的原始文字一般选择常用汉字,经过随机排列之后只能一个字一个字的输入,对参赛者来说是相对公平的方案。

在文字录入比赛(打字比赛)中,最公平的比赛用文本就是随机文本,这个随机汉字生成器便是为此所作。普通人的汉字录入速度一般是每分钟几十个到一百多个,我们可以生成一两千字的随机汉字文本,让比赛者录入完这些汉字,依据他们的比赛用时和正确率判断名次。生成随机汉字的原始文字一般选择常用汉字,经过随机排列之后只能一个字一个字的输入,对参赛者来说是相对公平的方案。

在文字录入比赛(打字比赛)中,最公平的比赛用文本就是随机文本,这个随机汉字生成器便是为此所作。普通人的汉字录入速度一般是每分钟几十个到一百多个,我们可以生成一两千字的随机汉字文本,让比赛者录入完这些汉字,依据他们的比赛用时和正确率判断名次。生成随机汉字的原始文字一般选择常用汉字,经过随机排列之后只能一个字一个字的输入,对参赛者来说是相对公平的方案。

\subsection{subsection 2}

在文字录入比赛(打字比赛)中,最公平的比赛用文本就是随机文本,这个随机汉字生成器便是为此所作。普通人的汉字录入速度一般是每分钟几十个到一百多个,我们可以生成一两千字的随机汉字文本,让比赛者录入完这些汉字,依据他们的比赛用时和正确率判断名次。生成随机汉字的原始文字一般选择常用汉字,经过随机排列之后只能一个字一个字的输入,对参赛者来说是相对公平的方案。

在文字录入比赛(打字比赛)中,最公平的比赛用文本就是随机文本,这个随机汉字生成器便是为此所作。普通人的汉字录入速度一般是每分钟几十个到一百多个,我们可以生成一两千字的随机汉字文本,让比赛者录入完这些汉字,依据他们的比赛用时和正确率判断名次。生成随机汉字的原始文字一般选择常用汉字,经过随机排列之后只能一个字一个字的输入,对参赛者来说是相对公平的方案。

\cleardoublepage
\chapter{chapter 3}

\section{section 1}

\subsection{subsection 1}

在文字录入比赛(打字比赛)中,最公平的比赛用文本就是随机文本,这个随机汉字生成器便是为此所作。普通人的汉字录入速度一般是每分钟几十个到一百多个,我们可以生成一两千字的随机汉字文本,让比赛者录入完这些汉字,依据他们的比赛用时和正确率判断名次。生成随机汉字的原始文字一般选择常用汉字,经过随机排列之后只能一个字一个字的输入,对参赛者来说是相对公平的方案。

在文字录入比赛(打字比赛)中,最公平的比赛用文本就是随机文本,这个随机汉字生成器便是为此所作。普通人的汉字录入速度一般是每分钟几十个到一百多个,我们可以生成一两千字的随机汉字文本,让比赛者录入完这些汉字,依据他们的比赛用时和正确率判断名次。生成随机汉字的原始文字一般选择常用汉字,经过随机排列之后只能一个字一个字的输入,对参赛者来说是相对公平的方案。

在文字录入比赛(打字比赛)中,最公平的比赛用文本就是随机文本,这个随机汉字生成器便是为此所作。普通人的汉字录入速度一般是每分钟几十个到一百多个,我们可以生成一两千字的随机汉字文本,让比赛者录入完这些汉字,依据他们的比赛用时和正确率判断名次。生成随机汉字的原始文字一般选择常用汉字,经过随机排列之后只能一个字一个字的输入,对参赛者来说是相对公平的方案。

\subsection{subsection 2}

在文字录入比赛(打字比赛)中,最公平的比赛用文本就是随机文本,这个随机汉字生成器便是为此所作。普通人的汉字录入速度一般是每分钟几十个到一百多个,我们可以生成一两千字的随机汉字文本,让比赛者录入完这些汉字,依据他们的比赛用时和正确率判断名次。生成随机汉字的原始文字一般选择常用汉字,经过随机排列之后只能一个字一个字的输入,对参赛者来说是相对公平的方案。

在文字录入比赛(打字比赛)中,最公平的比赛用文本就是随机文本,这个随机汉字生成器便是为此所作。普通人的汉字录入速度一般是每分钟几十个到一百多个,我们可以生成一两千字的随机汉字文本,让比赛者录入完这些汉字,依据他们的比赛用时和正确率判断名次。生成随机汉字的原始文字一般选择常用汉字,经过随机排列之后只能一个字一个字的输入,对参赛者来说是相对公平的方案。

在文字录入比赛(打字比赛)中,最公平的比赛用文本就是随机文本,这个随机汉字生成器便是为此所作。普通人的汉字录入速度一般是每分钟几十个到一百多个,我们可以生成一两千字的随机汉字文本,让比赛者录入完这些汉字,依据他们的比赛用时和正确率判断名次。生成随机汉字的原始文字一般选择常用汉字,经过随机排列之后只能一个字一个字的输入,对参赛者来说是相对公平的方案。

在文字录入比赛(打字比赛)中,最公平的比赛用文本就是随机文本,这个随机汉字生成器便是为此所作。普通人的汉字录入速度一般是每分钟几十个到一百多个,我们可以生成一两千字的随机汉字文本,让比赛者录入完这些汉字,依据他们的比赛用时和正确率判断名次。生成随机汉字的原始文字一般选择常用汉字,经过随机排列之后只能一个字一个字的输入,对参赛者来说是相对公平的方案。

\section{section 2}

在文字录入比赛(打字比赛)中,最公平的比赛用文本就是随机文本,这个随机汉字生成器便是为此所作。普通人的汉字录入速度一般是每分钟几十个到一百多个,我们可以生成一两千字的随机汉字文本,让比赛者录入完这些汉字,依据他们的比赛用时和正确率判断名次。生成随机汉字的原始文字一般选择常用汉字,经过随机排列之后只能一个字一个字的输入,对参赛者来说是相对公平的方案。

在文字录入比赛(打字比赛)中,最公平的比赛用文本就是随机文本,这个随机汉字生成器便是为此所作。普通人的汉字录入速度一般是每分钟几十个到一百多个,我们可以生成一两千字的随机汉字文本,让比赛者录入完这些汉字,依据他们的比赛用时和正确率判断名次。生成随机汉字的原始文字一般选择常用汉字,经过随机排列之后只能一个字一个字的输入,对参赛者来说是相对公平的方案。

\cleardoublepage
\chapter{chapter 4}

\section{section 1}

在文字录入比赛(打字比赛)中,最公平的比赛用文本就是随机文本,这个随机汉字生成器便是为此所作。普通人的汉字录入速度一般是每分钟几十个到一百多个,我们可以生成一两千字的随机汉字文本,让比赛者录入完这些汉字,依据他们的比赛用时和正确率判断名次。生成随机汉字的原始文字一般选择常用汉字,经过随机排列之后只能一个字一个字的输入,对参赛者来说是相对公平的方案。

在文字录入比赛(打字比赛)中,最公平的比赛用文本就是随机文本,这个随机汉字生成器便是为此所作。普通人的汉字录入速度一般是每分钟几十个到一百多个,我们可以生成一两千字的随机汉字文本,让比赛者录入完这些汉字,依据他们的比赛用时和正确率判断名次。生成随机汉字的原始文字一般选择常用汉字,经过随机排列之后只能一个字一个字的输入,对参赛者来说是相对公平的方案。

在文字录入比赛(打字比赛)中,最公平的比赛用文本就是随机文本,这个随机汉字生成器便是为此所作。普通人的汉字录入速度一般是每分钟几十个到一百多个,我们可以生成一两千字的随机汉字文本,让比赛者录入完这些汉字,依据他们的比赛用时和正确率判断名次。生成随机汉字的原始文字一般选择常用汉字,经过随机排列之后只能一个字一个字的输入,对参赛者来说是相对公平的方案。

在文字录入比赛(打字比赛)中,最公平的比赛用文本就是随机文本,这个随机汉字生成器便是为此所作。普通人的汉字录入速度一般是每分钟几十个到一百多个,我们可以生成一两千字的随机汉字文本,让比赛者录入完这些汉字,依据他们的比赛用时和正确率判断名次。生成随机汉字的原始文字一般选择常用汉字,经过随机排列之后只能一个字一个字的输入,对参赛者来说是相对公平的方案。

在文字录入比赛(打字比赛)中,最公平的比赛用文本就是随机文本,这个随机汉字生成器便是为此所作。普通人的汉字录入速度一般是每分钟几十个到一百多个,我们可以生成一两千字的随机汉字文本,让比赛者录入完这些汉字,依据他们的比赛用时和正确率判断名次。生成随机汉字的原始文字一般选择常用汉字,经过随机排列之后只能一个字一个字的输入,对参赛者来说是相对公平的方案。

在文字录入比赛(打字比赛)中,最公平的比赛用文本就是随机文本,这个随机汉字生成器便是为此所作。普通人的汉字录入速度一般是每分钟几十个到一百多个,我们可以生成一两千字的随机汉字文本,让比赛者录入完这些汉字,依据他们的比赛用时和正确率判断名次。生成随机汉字的原始文字一般选择常用汉字,经过随机排列之后只能一个字一个字的输入,对参赛者来说是相对公平的方案。

在文字录入比赛(打字比赛)中,最公平的比赛用文本就是随机文本,这个随机汉字生成器便是为此所作。普通人的汉字录入速度一般是每分钟几十个到一百多个,我们可以生成一两千字的随机汉字文本,让比赛者录入完这些汉字,依据他们的比赛用时和正确率判断名次。生成随机汉字的原始文字一般选择常用汉字,经过随机排列之后只能一个字一个字的输入,对参赛者来说是相对公平的方案。

\section{section 2}

在文字录入比赛(打字比赛)中,最公平的比赛用文本就是随机文本,这个随机汉字生成器便是为此所作。普通人的汉字录入速度一般是每分钟几十个到一百多个,我们可以生成一两千字的随机汉字文本,让比赛者录入完这些汉字,依据他们的比赛用时和正确率判断名次。生成随机汉字的原始文字一般选择常用汉字,经过随机排列之后只能一个字一个字的输入,对参赛者来说是相对公平的方案。

在文字录入比赛(打字比赛)中,最公平的比赛用文本就是随机文本,这个随机汉字生成器便是为此所作。普通人的汉字录入速度一般是每分钟几十个到一百多个,我们可以生成一两千字的随机汉字文本,让比赛者录入完这些汉字,依据他们的比赛用时和正确率判断名次。生成随机汉字的原始文字一般选择常用汉字,经过随机排列之后只能一个字一个字的输入,对参赛者来说是相对公平的方案。

\backmatter


\end{document}






























