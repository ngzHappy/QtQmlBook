
%用于快速测试

\documentclass[12pt,hyperref,UTF8]{ctexbook}

\let\counterwithout\relax
\let\counterwithin\relax

%@P143
\usepackage{xcolor}        %导入包xcolor
\usepackage[
a4paper ,
left=2.0cm,                %靠近装订线的边距
right=2.0cm,               %远离装订线的边距
top=2.0cm,
bottom=2.0cm,
headheight=1.3cm,
headsep=0.5cm,
marginparsep=0.1cm,
marginparwidth=0.1cm
]{geometry}                %导入包geometry

 %导入常用包
\usepackage{graphicx}
\usepackage{float}
\usepackage{amsmath}
\usepackage{cite}
\usepackage{caption}
\usepackage{titlesec}
\usepackage{chngcntr}
\usepackage{setspace}
\usepackage{tocbibind}  %设置目录
\usepackage{tocloft}
\usepackage{multicol}
\usepackage{listings}%引入程序代码

%%常见符号
\usepackage{wasysym}    %
\usepackage{textcomp}   %
\usepackage{pifont}     %

\usepackage{color}
\definecolor{colorbackgroundthisproject}{rgb}{1,1,1} %页面背景颜色
\definecolor{colortextthisproject}{rgb}{0,0,0}       %文字颜色
%设置页面颜色
\pagecolor{colorbackgroundthisproject}
%设置字体颜色
\color{colortextthisproject}

\usepackage{xhfill}
\usepackage{times}


%设置输出pdf格式
\usepackage[
    colorlinks=true ,
    %bookmarks=true,
    %bookmarksopen=false,
    %pdfpagemode=FullScreen,
    %pdfstartview=Fit,
    bookmarksnumbered=true,
    pdftitle={Qml} ,       %标题
    pdfauthor={Qml} ,      %作者
    pdfsubject={Qml} ,     %主题
    pdfkeywords={Qml} ,    %关键字
    linkcolor=colortextthisproject
]{hyperref}




\title{这是一本书}                              %书籍名称
\author{Good Luck}                              %作者名

%设置ctex
%P135
\CTEXsetup[ number={ \arabic{chapter} } ]{chapter}
%\CTEXsetup[ number={ \arabic{section} } , name={第,节} ]{section}

\begin{document}


\frontmatter
\pagestyle{empty}                 %关闭页眉页脚
\maketitle                        %生成封面

\cleardoublepage
\pagestyle{headings}              %开启页眉页脚
\pagenumbering{roman}             %重新开始页码编号
\tableofcontents                  %生成目录

\mainmatter
\cleardoublepage
%
%    arabic - 阿拉伯数字
%    roman - 小写的罗马数字
%    Roman - 大写的罗马数字
%    alph - 小写的字符形式
%    Alph -大写的字符形式
%
\pagenumbering{arabic}  %重新开始页码编号
\chapter{chapter 1}
%P129
\section{section 1}
\label{testsection}

\subsection{subsection 1}

在文字录入比赛(打字比赛)中,最公平的比赛用文本就是随机文本,这个随机汉字生成器便是为此所作。普通人的汉字录入速度一般是每分钟几十个到一百多个,我们可以生成一两千字的随机汉字文本,让比赛者录入完这些汉字,依据他们的比赛用时和正确率判断名次。生成随机汉字的原始文字一般选择常用汉字,经过随机排列之后只能一个字一个字的输入,对参赛者来说是相对公平的方案。

在文字录入A\ref{testsection}A\pageref{testsection}A比赛(打字比赛)中,最公平的比赛用文本就是随机文本,这个随机汉字生成器便是为此所作。普通人的汉字录入速度一般是每分钟几十个到一百多个,我们可以生成一两千字的随机汉字文本,让比赛者录入完这些汉字,依据他们的比赛用时和正确率判断名次。生成随机汉字的原始文字一般选择常用汉字,经过随机排列之后只能一个字一个字的输入,对参赛者来说是相对公平的方案。

\subsection{subsection 2}

在文字录入比赛(打字比赛)中,最公平的比赛用文本就是随机文本,这个随机汉字生成器便是为此所作。普通人的汉字录入速度一般是每分钟几十个到一百多个,我们可以生成一两千字的随机汉字文本,让比赛者录入完这些汉字,依据他们的比赛用时和正确率判断名次。生成随机汉字的原始文字一般选择常用汉字,经过随机排列之后只能一个字一个字的输入,对参赛者来说是相对公平的方案。

在文字录入比赛(打字比赛)中,最公平的比赛用文本就是随机文本,这个随机汉字生成器便是为此所作。普通人的汉字录入速度一般是每分钟几十个到一百多个,我们可以生成一两千字的随机汉字文本,让比赛者录入完这些汉字,依据他们的比赛用时和正确率判断名次。生成随机汉字的原始文字一般选择常用汉字,经过随机排列之后只能一个字一个字的输入,对参赛者来说是相对公平的方案。

\section{section 2}

\subsection{subsection 1}

在文字录入比赛(打字比赛)中,最公平的比赛用文本就是随机文本,这个随机汉字生成器便是为此所作。普通人的汉字录入速度一般是每分钟几十个到一百多个,我们可以生成一两千字的随机汉字文本,让比赛者录入完这些汉字,依据他们的比赛用时和正确率判断名次。生成随机汉字的原始文字一般选择常用汉字,经过随机排列之后只能一个字一个字的输入,对参赛者来说是相对公平的方案。

在文字录入比赛(打字比赛)中,最公平的比赛用文本就是随机文本,这个随机汉字生成器便是为此所作。普通人的汉字录入速度一般是每分钟几十个到一百多个,我们可以生成一两千字的随机汉字文本,让比赛者录入完这些汉字,依据他们的比赛用时和正确率判断名次。生成随机汉字的原始文字一般选择常用汉字,经过随机排列之后只能一个字一个字的输入,对参赛者来说是相对公平的方案。

在文字录入比赛(打字比赛)中,最公平的比赛用文本就是随机文本,这个随机汉字生成器便是为此所作。普通人的汉字录入速度一般是每分钟几十个到一百多个,我们可以生成一两千字的随机汉字文本,让比赛者录入完这些汉字,依据他们的比赛用时和正确率判断名次。生成随机汉字的原始文字一般选择常用汉字,经过随机排列之后只能一个字一个字的输入,对参赛者来说是相对公平的方案。

\subsection{subsection 2}

在文字录入比赛(打字比赛)中,最公平的比赛用文本就是随机文本,这个随机汉字生成器便是为此所作。普通人的汉字录入速度一般是每分钟几十个到一百多个,我们可以生成一两千字的随机汉字文本,让比赛者录入完这些汉字,依据他们的比赛用时和正确率判断名次。生成随机汉字的原始文字一般选择常用汉字,经过随机排列之后只能一个字一个字的输入,对参赛者来说是相对公平的方案。

在文字录入比赛(打字比赛)中,最公平的比赛用文本就是随机文本,这个随机汉字生成器便是为此所作。普通人的汉字录入速度一般是每分钟几十个到一百多个,我们可以生成一两千字的随机汉字文本,让比赛者录入完这些汉字,依据他们的比赛用时和正确率判断名次。生成随机汉字的原始文字一般选择常用汉字,经过随机排列之后只能一个字一个字的输入,对参赛者来说是相对公平的方案。


\cleardoublepage
\chapter{chapter 2}

\section{section 1}

\subsection{subsection 1}

在文字录入比赛(打字比赛)中,最公平的比赛用文本就是随机文本,这个随机汉字生成器便是为此所作。普通人的汉字录入速度一般是每分钟几十个到一百多个,我们可以生成一两千字的随机汉字文本,让比赛者录入完这些汉字,依据他们的比赛用时和正确率判断名次。生成随机汉字的原始文字一般选择常用汉字,经过随机排列之后只能一个字一个字的输入,对参赛者来说是相对公平的方案。

在文字录入比赛(打字比赛)中,最公平的比赛用文本就是随机文本,这个随机汉字生成器便是为此所作。普通人的汉字录入速度一般是每分钟几十个到一百多个,我们可以生成一两千字的随机汉字文本,让比赛者录入完这些汉字,依据他们的比赛用时和正确率判断名次。生成随机汉字的原始文字一般选择常用汉字,经过随机排列之后只能一个字一个字的输入,对参赛者来说是相对公平的方案。

在文字录入比赛(打字比赛)中,最公平的比赛用文本就是随机文本,这个随机汉字生成器便是为此所作。普通人的汉字录入速度一般是每分钟几十个到一百多个,我们可以生成一两千字的随机汉字文本,让比赛者录入完这些汉字,依据他们的比赛用时和正确率判断名次。生成随机汉字的原始文字一般选择常用汉字,经过随机排列之后只能一个字一个字的输入,对参赛者来说是相对公平的方案。

在文字录入比赛(打字比赛)中,最公平的比赛用文本就是随机文本,这个随机汉字生成器便是为此所作。普通人的汉字录入速度一般是每分钟几十个到一百多个,我们可以生成一两千字的随机汉字文本,让比赛者录入完这些汉字,依据他们的比赛用时和正确率判断名次。生成随机汉字的原始文字一般选择常用汉字,经过随机排列之后只能一个字一个字的输入,对参赛者来说是相对公平的方案。

在文字录入比赛(打字比赛)中,最公平的比赛用文本就是随机文本,这个随机汉字生成器便是为此所作。普通人的汉字录入速度一般是每分钟几十个到一百多个,我们可以生成一两千字的随机汉字文本,让比赛者录入完这些汉字,依据他们的比赛用时和正确率判断名次。生成随机汉字的原始文字一般选择常用汉字,经过随机排列之后只能一个字一个字的输入,对参赛者来说是相对公平的方案。

在文字录入比赛(打字比赛)中,最公平的比赛用文本就是随机文本,这个随机汉字生成器便是为此所作。普通人的汉字录入速度一般是每分钟几十个到一百多个,我们可以生成一两千字的随机汉字文本,让比赛者录入完这些汉字,依据他们的比赛用时和正确率判断名次。生成随机汉字的原始文字一般选择常用汉字,经过随机排列之后只能一个字一个字的输入,对参赛者来说是相对公平的方案。

在文字录入比赛(打字比赛)中,最公平的比赛用文本就是随机文本,这个随机汉字生成器便是为此所作。普通人的汉字录入速度一般是每分钟几十个到一百多个,我们可以生成一两千字的随机汉字文本,让比赛者录入完这些汉字,依据他们的比赛用时和正确率判断名次。生成随机汉字的原始文字一般选择常用汉字,经过随机排列之后只能一个字一个字的输入,对参赛者来说是相对公平的方案。

\subsection{subsection 2}

在文字录入比赛(打字比赛)中,最公平的比赛用文本就是随机文本,这个随机汉字生成器便是为此所作。普通人的汉字录入速度一般是每分钟几十个到一百多个,我们可以生成一两千字的随机汉字文本,让比赛者录入完这些汉字,依据他们的比赛用时和正确率判断名次。生成随机汉字的原始文字一般选择常用汉字,经过随机排列之后只能一个字一个字的输入,对参赛者来说是相对公平的方案。

在文字录入比赛(打字比赛)中,最公平的比赛用文本就是随机文本,这个随机汉字生成器便是为此所作。普通人的汉字录入速度一般是每分钟几十个到一百多个,我们可以生成一两千字的随机汉字文本,让比赛者录入完这些汉字,依据他们的比赛用时和正确率判断名次。生成随机汉字的原始文字一般选择常用汉字,经过随机排列之后只能一个字一个字的输入,对参赛者来说是相对公平的方案。

\cleardoublepage
\chapter{chapter 3}

\section{section 1}

\subsection{subsection 1}

在文字录入比赛(打字比赛)中,最公平的比赛用文本就是随机文本,这个随机汉字生成器便是为此所作。普通人的汉字录入速度一般是每分钟几十个到一百多个,我们可以生成一两千字的随机汉字文本,让比赛者录入完这些汉字,依据他们的比赛用时和正确率判断名次。生成随机汉字的原始文字一般选择常用汉字,经过随机排列之后只能一个字一个字的输入,对参赛者来说是相对公平的方案。

在文字录入比赛(打字比赛)中,最公平的比赛用文本就是随机文本,这个随机汉字生成器便是为此所作。普通人的汉字录入速度一般是每分钟几十个到一百多个,我们可以生成一两千字的随机汉字文本,让比赛者录入完这些汉字,依据他们的比赛用时和正确率判断名次。生成随机汉字的原始文字一般选择常用汉字,经过随机排列之后只能一个字一个字的输入,对参赛者来说是相对公平的方案。

在文字录入比赛(打字比赛)中,最公平的比赛用文本就是随机文本,这个随机汉字生成器便是为此所作。普通人的汉字录入速度一般是每分钟几十个到一百多个,我们可以生成一两千字的随机汉字文本,让比赛者录入完这些汉字,依据他们的比赛用时和正确率判断名次。生成随机汉字的原始文字一般选择常用汉字,经过随机排列之后只能一个字一个字的输入,对参赛者来说是相对公平的方案。

\subsection{subsection 2}

在文字录入比赛(打字比赛)中,最公平的比赛用文本就是随机文本,这个随机汉字生成器便是为此所作。普通人的汉字录入速度一般是每分钟几十个到一百多个,我们可以生成一两千字的随机汉字文本,让比赛者录入完这些汉字,依据他们的比赛用时和正确率判断名次。生成随机汉字的原始文字一般选择常用汉字,经过随机排列之后只能一个字一个字的输入,对参赛者来说是相对公平的方案。

在文字录入比赛(打字比赛)中,最公平的比赛用文本就是随机文本,这个随机汉字生成器便是为此所作。普通人的汉字录入速度一般是每分钟几十个到一百多个,我们可以生成一两千字的随机汉字文本,让比赛者录入完这些汉字,依据他们的比赛用时和正确率判断名次。生成随机汉字的原始文字一般选择常用汉字,经过随机排列之后只能一个字一个字的输入,对参赛者来说是相对公平的方案。

在文字录入比赛(打字比赛)中,最公平的比赛用文本就是随机文本,这个随机汉字生成器便是为此所作。普通人的汉字录入速度一般是每分钟几十个到一百多个,我们可以生成一两千字的随机汉字文本,让比赛者录入完这些汉字,依据他们的比赛用时和正确率判断名次。生成随机汉字的原始文字一般选择常用汉字,经过随机排列之后只能一个字一个字的输入,对参赛者来说是相对公平的方案。

在文字录入比赛(打字比赛)中,最公平的比赛用文本就是随机文本,这个随机汉字生成器便是为此所作。普通人的汉字录入速度一般是每分钟几十个到一百多个,我们可以生成一两千字的随机汉字文本,让比赛者录入完这些汉字,依据他们的比赛用时和正确率判断名次。生成随机汉字的原始文字一般选择常用汉字,经过随机排列之后只能一个字一个字的输入,对参赛者来说是相对公平的方案。

\section{section 2}

在文字录入比赛(打字比赛)中,最公平的比赛用文本就是随机文本,这个随机汉字生成器便是为此所作。普通人的汉字录入速度一般是每分钟几十个到一百多个,我们可以生成一两千字的随机汉字文本,让比赛者录入完这些汉字,依据他们的比赛用时和正确率判断名次。生成随机汉字的原始文字一般选择常用汉字,经过随机排列之后只能一个字一个字的输入,对参赛者来说是相对公平的方案。

在文字录入比赛(打字比赛)中,最公平的比赛用文本就是随机文本,这个随机汉字生成器便是为此所作。普通人的汉字录入速度一般是每分钟几十个到一百多个,我们可以生成一两千字的随机汉字文本,让比赛者录入完这些汉字,依据他们的比赛用时和正确率判断名次。生成随机汉字的原始文字一般选择常用汉字,经过随机排列之后只能一个字一个字的输入,对参赛者来说是相对公平的方案。

\cleardoublepage
\chapter{chapter 4}

\section{section 1}

在文字录入比赛(打字比赛)中,最公平的比赛用文本就是随机文本,这个随机汉字生成器便是为此所作。普通人的汉字录入速度一般是每分钟几十个到一百多个,我们可以生成一两千字的随机汉字文本,让比赛者录入完这些汉字,依据他们的比赛用时和正确率判断名次。生成随机汉字的原始文字一般选择常用汉字,经过随机排列之后只能一个字一个字的输入,对参赛者来说是相对公平的方案。

在文字录入比赛(打字比赛)中,最公平的比赛用文本就是随机文本,这个随机汉字生成器便是为此所作。普通人的汉字录入速度一般是每分钟几十个到一百多个,我们可以生成一两千字的随机汉字文本,让比赛者录入完这些汉字,依据他们的比赛用时和正确率判断名次。生成随机汉字的原始文字一般选择常用汉字,经过随机排列之后只能一个字一个字的输入,对参赛者来说是相对公平的方案。

在文字录入比赛(打字比赛)中,最公平的比赛用文本就是随机文本,这个随机汉字生成器便是为此所作。普通人的汉字录入速度一般是每分钟几十个到一百多个,我们可以生成一两千字的随机汉字文本,让比赛者录入完这些汉字,依据他们的比赛用时和正确率判断名次。生成随机汉字的原始文字一般选择常用汉字,经过随机排列之后只能一个字一个字的输入,对参赛者来说是相对公平的方案。

在文字录入比赛(打字比赛)中,最公平的比赛用文本就是随机文本,这个随机汉字生成器便是为此所作。普通人的汉字录入速度一般是每分钟几十个到一百多个,我们可以生成一两千字的随机汉字文本,让比赛者录入完这些汉字,依据他们的比赛用时和正确率判断名次。生成随机汉字的原始文字一般选择常用汉字,经过随机排列之后只能一个字一个字的输入,对参赛者来说是相对公平的方案。

在文字录入比赛(打字比赛)中,最公平的比赛用文本就是随机文本,这个随机汉字生成器便是为此所作。普通人的汉字录入速度一般是每分钟几十个到一百多个,我们可以生成一两千字的随机汉字文本,让比赛者录入完这些汉字,依据他们的比赛用时和正确率判断名次。生成随机汉字的原始文字一般选择常用汉字,经过随机排列之后只能一个字一个字的输入,对参赛者来说是相对公平的方案。

在文字录入比赛(打字比赛)中,最公平的比赛用文本就是随机文本,这个随机汉字生成器便是为此所作。普通人的汉字录入速度一般是每分钟几十个到一百多个,我们可以生成一两千字的随机汉字文本,让比赛者录入完这些汉字,依据他们的比赛用时和正确率判断名次。生成随机汉字的原始文字一般选择常用汉字,经过随机排列之后只能一个字一个字的输入,对参赛者来说是相对公平的方案。

在文字录入比赛(打字比赛)中,最公平的比赛用文本就是随机文本,这个随机汉字生成器便是为此所作。普通人的汉字录入速度一般是每分钟几十个到一百多个,我们可以生成一两千字的随机汉字文本,让比赛者录入完这些汉字,依据他们的比赛用时和正确率判断名次。生成随机汉字的原始文字一般选择常用汉字,经过随机排列之后只能一个字一个字的输入,对参赛者来说是相对公平的方案。

\section{section 2}

在文字录入比赛(打字比赛)中,最公平的比赛用文本就是随机文本,这个随机汉字生成器便是为此所作。普通人的汉字录入速度一般是每分钟几十个到一百多个,我们可以生成一两千字的随机汉字文本,让比赛者录入完这些汉字,依据他们的比赛用时和正确率判断名次。生成随机汉字的原始文字一般选择常用汉字,经过随机排列之后只能一个字一个字的输入,对参赛者来说是相对公平的方案。

在文字录入比赛(打字比赛)中,最公平的比赛用文本就是随机文本,这个随机汉字生成器便是为此所作。普通人的汉字录入速度一般是每分钟几十个到一百多个,我们可以生成一两千字的随机汉字文本,让比赛者录入完这些汉字,依据他们的比赛用时和正确率判断名次。生成随机汉字的原始文字一般选择常用汉字,经过随机排列之后只能一个字一个字的输入,对参赛者来说是相对公平的方案。

\backmatter


\end{document}






























