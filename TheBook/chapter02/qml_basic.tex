
%使用XeLaTeX编译
%版权所有,翻版必究
%本文件由程序自动生成,任何修改将被覆盖
%2019 年 01 月 23 日




\FloatBarrier
\section{
QML语法
}\label{c000011s000000}


%编码
\FloatBarrier
\subsection{
文件编码
}\label{c000011s000000s03}


自从Qt 5开始,Qt所有源代码只接受一种编码,
那就是UTF\hspace{0.05em}\rule[0.7ex]{0.4em}{0.65pt}\hspace{0.05em}8。

无论是C{\sourcefonttwo{}+}{\sourcefonttwo{}+}、QML、JavaScript或者是qmake工程文件,
都应当只采用UTF\hspace{0.05em}\rule[0.7ex]{0.4em}{0.65pt}\hspace{0.05em}8编码。

特别的是,有些UTF\hspace{0.05em}\rule[0.7ex]{0.4em}{0.65pt}\hspace{0.05em}8文件开头会有三个特殊字符:
\begin{littlelongworld}0xEF0xBB0xBF
\end{littlelongworld}以以上三个特殊字符开头的UTF\hspace{0.05em}\rule[0.7ex]{0.4em}{0.65pt}\hspace{0.05em}8文件一般
被称为UTF\hspace{0.05em}\rule[0.7ex]{0.4em}{0.65pt}\hspace{0.05em}8 with BOM。

Qt集成开发环境所有编译器都识别BOM,除了qmake。

也就是读者在书写qmake工程文件(\raisebox{-0.35ex}{\sourcefonttwo{}*}.pro或者\raisebox{-0.35ex}{\sourcefonttwo{}*}.pri文件)时
一定不能给文件添加BOM。实际上,qmake只识别ASCII码的
127个字符。不应当在qmake工程文件中使用超出ASCII码的
字符。

但是在书写C{\sourcefonttwo{}+}{\sourcefonttwo{}+}、QML、JavaScript文件时,可以给文件添加BOM。

在不同的操作系统下,系统的默认文件编码会有所不同。
由于历史原因和编程习惯,一些计算机语言不要求在源码中指定源码的
编码。
这就会造成,如果不给这些计算机语言的源码添加BOM,
编译器会错误的识别源代码编码,
造成编译错误和运行逻辑错误。

对于这类计算机语言,最好对它们的源码添加BOM。

而对于一些现代计算机语言比如Html,
一般的在源码中就指定了源码的编码。
因而,对于这些计算机语言,最好不要对它们的源码添加BOM。

本书对C{\sourcefonttwo{}+}{\sourcefonttwo{}+}源文件,QML源文件,以及JavaScript源文件添加BOM
以最大程度降低平台差异性。

%注释
\FloatBarrier
\subsection{
注释
}\label{c000011s000000s02}







%使用XeLaTeX编译
%版权所有,翻版必究
%本文件由程序自动生成,任何修改将被覆盖
%2019 年 01 月 23 日




\FloatBarrier
\subsection{
属性
}\label{c000011s000000s01}


%QML Basic Types
%bool
%double
%enumeration
%int
%list
%real
%string
%url
%var
%date
%point
%rect
%size 

\FloatBarrier
\subsubsection{
基本类型
}\label{c000011s000000s01s01}




%使用XeLaTeX编译
%版权所有,翻版必究
%本文件由程序自动生成,任何修改将被覆盖
%2019 年 01 月 23 日



%begin表
\FloatBarrier                                  %强制完成浮动体布局
\begin{longtable}{cc}

%表头....
\toprule{}名称
&
描述[公式]%there must use marginnote ...
\marginnote{\setlength\fboxsep{2pt}\fbox{\footnotesize{\kaishu\tablename\,}\footnotesize{\ref{tb000002}}}}
\\ \midrule 
\endfirsthead

%表尾...
\bottomrule
\caption{混合类型}\label{tb000002} 
\endlastfoot

%重复表头
\toprule{}名称
&
描述[公式]
\\ \midrule
\endhead
%重复表尾
\midrule
\endfoot 
normal
    &
使用Alpha通道进行混合   
    \\

addition
    &
结果是source与foregroundSource的和
    \\

average
    &
结果是source与foregroundSource的均值
    \\

color
    &
结果是source的亮度值,
foregroundSource的色调与饱和度
    \\

colorBurn
    &
颜色加深$\left[v=1-(1-b)/a\right]$
 %与Photoshop colorBurn效果一致
    \\

colorDodge
    &
颜色减淡 $\left[v=b/(1-a)\right]$
 %与Photoshop colorDodge效果一致
    \\

darken
    &
变暗$\left[v=min(a,b)\right]$
 %与Photoshop  效果一致
    \\

darkerColor
    &
深色$\left[v=\begin{cases}
a, & a_r+a_r+a_b<b_r+b_g+b_b \\ 
b, & a_r+a_r+a_b>b_r+b_g+b_b
\end{cases}\right]$
 %与Photoshop  效果一致
    \\

difference
    &
    &
变暗$\left[v=abs(b-a)\right]$
 %与Photoshop  效果一致
    \\

divide
    &
    &
划分$\left[v=b/a\right]$
 %与Photoshop  效果一致
    \\

exclusion
    &
    &
排除$\left[v=b/a\right]$
 %与Photoshop  效果一致
    \\
\end{longtable}
%end表





%使用XeLaTeX编译
%版权所有,翻版必究
%本文件由程序自动生成,任何修改将被覆盖
%2019 年 01 月 23 日

























%使用XeLaTeX编译
%版权所有,翻版必究
%本文件由程序自动生成,任何修改将被覆盖
%2019 年 01 月 23 日





























%使用XeLaTeX编译
%版权所有,翻版必究
%本文件由程序自动生成,任何修改将被覆盖
%2019 年 01 月 23 日



