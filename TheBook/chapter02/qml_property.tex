
%使用XeLaTeX编译
%版权所有,翻版必究
%本文件由程序自动生成,任何修改将被覆盖
%2019 年 01 月 23 日




\FloatBarrier
\subsection{
类型与对象属性
}\label{c000011s000000s01}


QML被设计为一门可以支持JavaScript的
弱类型语言。

虽然绝大多数JavaScript代码可以在QML
环境之中正常运行。
但并不意味着JavaScript语义与QML语义
完全一致。

一般而言,QML对JavaScript做了适当扩展。

如果读者完全按照JavaScript语义理解
QML类型系统,倒也不算错。
但损失大量效率是免不了的。

QML类型可以分为
基本类型、
扩展类型
和用户自定义类型。

%QML Basic Types
%bool
%double
%enumeration
%int
%list
%real
%string
%url
%var
%date
%point
%rect
%size


%基本类型
\FloatBarrier
\subsubsection{
基本类型
}\label{c000011s000000s01s01}


%http://doc.qt.io/qt-5/qml-int.html

\tablename\ \ref{tb000005}展示了
QML中的基本类型。


%使用XeLaTeX编译
%版权所有,翻版必究
%本文件由程序自动生成,任何修改将被覆盖
%2019 年 01 月 23 日



%begin表
\FloatBarrier                                  %强制完成浮动体布局
\begin{longtable}{cc}

%表头....
\toprule{}名称
&
描述[公式]%there must use marginnote ...
\marginnote{\setlength\fboxsep{2pt}\fbox{\footnotesize{\kaishu\tablename\,}\footnotesize{\ref{tb000002}}}}
\\ \midrule 
\endfirsthead

%表尾...
\bottomrule
\caption{混合类型}\label{tb000002} 
\endlastfoot

%重复表头
\toprule{}名称
&
描述[公式]
\\ \midrule
\endhead
%重复表尾
\midrule
\endfoot 
normal
    &
使用Alpha通道进行混合   
    \\

addition
    &
结果是source与foregroundSource的和
    \\

average
    &
结果是source与foregroundSource的均值
    \\

color
    &
结果是source的亮度值,
foregroundSource的色调与饱和度
    \\

colorBurn
    &
颜色加深$\left[v=1-(1-b)/a\right]$
 %与Photoshop colorBurn效果一致
    \\

colorDodge
    &
颜色减淡 $\left[v=b/(1-a)\right]$
 %与Photoshop colorDodge效果一致
    \\

darken
    &
变暗$\left[v=min(a,b)\right]$
 %与Photoshop  效果一致
    \\

darkerColor
    &
深色$\left[v=\begin{cases}
a, & a_r+a_r+a_b<b_r+b_g+b_b \\ 
b, & a_r+a_r+a_b>b_r+b_g+b_b
\end{cases}\right]$
 %与Photoshop  效果一致
    \\

difference
    &
    &
变暗$\left[v=abs(b-a)\right]$
 %与Photoshop  效果一致
    \\

divide
    &
    &
划分$\left[v=b/a\right]$
 %与Photoshop  效果一致
    \\

exclusion
    &
    &
排除$\left[v=b/a\right]$
 %与Photoshop  效果一致
    \\
\end{longtable}
%end表





%使用XeLaTeX编译
%版权所有,翻版必究
%本文件由程序自动生成,任何修改将被覆盖
%2019 年 01 月 23 日





\tablename\ \ref{tb000005}中的基本类型
是直接内嵌在QML引擎中的。
Qt并没有提供接口可以让读者直接定义
像int,double这样的原生
类型。读者如果需要扩展QML类型,只能继承自QObject类或其子类。


    % Data Type Conversion Between QML and C++


\begin{itemize}

\item bool

QML中的布尔类型根绝大多数计算机语言一致,只有true和false两个值。
\item int

QML中的整型对应于C{\sourcefonttwo{}+}{\sourcefonttwo{}+}中的int,其安全使用范围是
\hspace{0.05em}\rule[0.7ex]{0.4em}{0.65pt}\hspace{0.05em}2000000000\raisebox{0.16ex}{\sourcefonttwo\~{}}2000000000。

很多时候应用程序需要的是int64,这时候应当使用
var。

\item double与real

QML中的double与real没有什么区别,对应于
IEEE 754标准中规定的64位双精度浮点数。
\item enumeration

QML中的枚举类型,既可来源于C{\sourcefonttwo{}+}{\sourcefonttwo{}+}的导出,
也可来源于QML中的定义。
\item url

QML中的路径都是url类型,它与QUrl一致。
\item string

QML中的字符串除了可以使用JavaScript中的所有方法之外,
还可以使用QString中的arg函数。
\item list

QML中list被设计用于包装QML对象,如果需要基本类型容器,应当使用var。
\item var

QML中的var用于代表一切合法类型,包括信号槽,容器,基本类型以及一切
可以被QVariant识别的类型……

在一些时候读者会发现variant这个词,它与var是一致的。
支持这个词仅仅是为了兼容老版本的QML。

\end{itemize}

%扩展类型
\FloatBarrier
\subsubsection{
扩展类型
}\label{c000011s000000s01s02}

除了第\ref{c000011s000000s01s01}节提到的
基本类型之外,
不同的QML库还提供了一些扩展类型。

对于一般读者而言,这些扩展类型和基本类型之间除了构造方法之外
没有什么不同。


%自定义类型
\FloatBarrier
\subsubsection{
自定义类型
}\label{c000011s000000s01s03}

QML被设计为一门专门用于扩展Qt C{\sourcefonttwo{}+}{\sourcefonttwo{}+}的语言,
因而,从Qt C{\sourcefonttwo{}+}{\sourcefonttwo{}+}向QML导出自定义类是相当自然的。
详细内容见第\ \ref{c000012}章。

%QML常用基本与Qt C++常用类型对应表
\FloatBarrier
\subsubsection{
QML常用基本与Qt C{\sourcefonttwo{}+}{\sourcefonttwo{}+}常用类型对应表
}\label{c000011s000000s01s04}


\tablename\ \ref{tb000005}展示了
QML常用基本与Qt C{\sourcefonttwo{}+}{\sourcefonttwo{}+}常用类型的对应关系。


%使用XeLaTeX编译
%版权所有,翻版必究
%本文件由程序自动生成,任何修改将被覆盖
%2019 年 01 月 23 日



%begin表
\FloatBarrier                                  %强制完成浮动体布局
\begin{longtable}{cc}

%表头....
\toprule{}名称
&
描述[公式]%there must use marginnote ...
\marginnote{\setlength\fboxsep{2pt}\fbox{\footnotesize{\kaishu\tablename\,}\footnotesize{\ref{tb000002}}}}
\\ \midrule 
\endfirsthead

%表尾...
\bottomrule
\caption{混合类型}\label{tb000002} 
\endlastfoot

%重复表头
\toprule{}名称
&
描述[公式]
\\ \midrule
\endhead
%重复表尾
\midrule
\endfoot 
normal
    &
使用Alpha通道进行混合   
    \\

addition
    &
结果是source与foregroundSource的和
    \\

average
    &
结果是source与foregroundSource的均值
    \\

color
    &
结果是source的亮度值,
foregroundSource的色调与饱和度
    \\

colorBurn
    &
颜色加深$\left[v=1-(1-b)/a\right]$
 %与Photoshop colorBurn效果一致
    \\

colorDodge
    &
颜色减淡 $\left[v=b/(1-a)\right]$
 %与Photoshop colorDodge效果一致
    \\

darken
    &
变暗$\left[v=min(a,b)\right]$
 %与Photoshop  效果一致
    \\

darkerColor
    &
深色$\left[v=\begin{cases}
a, & a_r+a_r+a_b<b_r+b_g+b_b \\ 
b, & a_r+a_r+a_b>b_r+b_g+b_b
\end{cases}\right]$
 %与Photoshop  效果一致
    \\

difference
    &
    &
变暗$\left[v=abs(b-a)\right]$
 %与Photoshop  效果一致
    \\

divide
    &
    &
划分$\left[v=b/a\right]$
 %与Photoshop  效果一致
    \\

exclusion
    &
    &
排除$\left[v=b/a\right]$
 %与Photoshop  效果一致
    \\
\end{longtable}
%end表





%使用XeLaTeX编译
%版权所有,翻版必究
%本文件由程序自动生成,任何修改将被覆盖
%2019 年 01 月 23 日





对于\tablename\ \ref{tb000005}中
没有提及的类型,一般用var代替即可。

%容器
\FloatBarrier
\subsubsection{
QML可以识别的Qt C{\sourcefonttwo{}+}{\sourcefonttwo{}+}容器
}\label{c000011s000000s01s05}


一般而言如果一个Qt C{\sourcefonttwo{}+}{\sourcefonttwo{}+}类型能够被QML识别,
那么用QList和QMap包装这个对象,也能够被QML识别。

比如,Qt C{\sourcefonttwo{}+}{\sourcefonttwo{}+}端
QUrl能够被
QML识别,那么
QList<QUrl>
也能被QML识别。

更加一般的是Qt C{\sourcefonttwo{}+}{\sourcefonttwo{}+}端的
QVariantList容器和QVariantMap容器,
总是能被QML端识别。
但比使用更加具体的对象容器效率要低。

除了可以使用QList和QMap
包装对象,也可以使用
QVector、
std::vector包装
对象。

值得注意的是,包装QString最好使用QStringList类,
包装QObject \raisebox{-0.35ex}{\sourcefonttwo{}*} 最好使用QObjectList类。


% ______all_key_words
% the_book_chapter the_book_subsection the_book_subsubsection
% the_book_section the_book_image the_book_table
% the_book_file the_book_tree_file the_book_command_file
% littlelongworld tabbing ref
% figurename tablename filesourcenumbernameone
% treeindexnumbernameone commandnumbernameone footnote
% item itemize comment textbullet
% \hspace*{\parindent}
% FloatBarrier







%使用XeLaTeX编译
%版权所有,翻版必究
%本文件由程序自动生成,任何修改将被覆盖
%2019 年 01 月 23 日



