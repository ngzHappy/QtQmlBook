
%使用XeLaTeX编译
%版权所有,翻版必究
%本文件由程序自动生成,任何修改将被覆盖
%2019 年 01 月 23 日



%begin表
\FloatBarrier                                  %强制完成浮动体布局
\begin{longtable}{ll}

%表头....
\toprule{}Qt C{\sourcefonttwo{}+}{\sourcefonttwo{}+}类型
&
QML类型%there must use marginnote ...
\marginnote{\setlength\fboxsep{2pt}\fbox{\footnotesize{\kaishu\tablename\,}\footnotesize{\ref{tb000006}}}}
\\ \midrule
\endfirsthead

%表尾...
\endlastfoot

%重复表头
\toprule{}Qt C{\sourcefonttwo{}+}{\sourcefonttwo{}+}类型
&
QML类型
\\ \midrule
\endhead
%重复表尾
\midrule
\endfoot
bool
    &
bool
    \\

unsigned int, int
    &
int
    \\

double
    &
double
    \\

float, qreal
    &
real
    \\

float, qreal
    &
real
    \\

QString
    &
string
    \\

QUrl
    &
url
    \\

QColor
    &
color
    \\

QFont
    &
font
    \\    

QDateTime
    &
date
    \\   

QPoint, QPointF
    &
point
    \\  

QSize, QSizeF
    &
size
    \\  

QRect, QRectF
    &
rect
    \\  

QMatrix4x4
    &
matrix4x4
    \\  

QQuaternion
    &
quaternion
    \\      

QVector2D, QVector3D, QVector4D
    &
vector2d, vector3d, vector4d
    \\   

Enums declared with Q\underline{\hspace{0.5em}}ENUM() or Q\underline{\hspace{0.5em}}ENUMS()
    &
enumeration
    \\
\bottomrule            %表底部线
\caption{QML常用基本与Qt C{\sourcefonttwo{}+}{\sourcefonttwo{}+}常用类型对应关系}\label{tb000006} %表标题
\end{longtable}
%end表





%使用XeLaTeX编译
%版权所有,翻版必究
%本文件由程序自动生成,任何修改将被覆盖
%2019 年 01 月 23 日



