%使用xelatex编译
%版权所有,翻版必究
%本文件由程序自动生成,任何修改将被覆盖




%


\subsection{
在Linux平台下搭建开发环境
}\label{s000210}


Linux有众多发行版,
如果读者是首次在Linux平台下搭建Qt开发环境,建
议读者使用Ubuntu等使用者较多的版本。


\subsubsection{
在Linux平台下安装Qt
}\label{ss000410}



如\commandnumbernameone\ \ref{command000000},所示。

\begin{itemize}
\item 第1行命令用于安装基本C{\sourcefonttwo{}+}{\sourcefonttwo{}+}开发环境;
\item 第2行命令用于安装OpenGL环境;
\end{itemize}



%\begin{spacing}{1.0}
\refstepcounter{commandnumber}\label{command000000}    %增加目录树编号
\begin{lstlisting}[caption=GoodLuck,
title=\commandnumbernameone \thecommandnumber
]
sudo apt-get install build-essential
sudo apt-get install libgl1-mesa-dev
\end{lstlisting}          %抄录环境
%\end{spacing}



在不同Linux发行版本这些命令有所不同,
即使是同一发行版本,
随着时间推移命令也会有所变化。

读者可以访问\ \url{https://wiki.qt.io/Main}
获得更加详细的帮助。

读者可以参照\ \ref{ss000110}
节相关内容下载最新的Qt开发包。





\subsubsection{
在Linux平台下安装Boost
}\label{ss000510}












%使用xelatex编译
%版权所有,翻版必究
%本文件由程序自动生成,任何修改将被覆盖



