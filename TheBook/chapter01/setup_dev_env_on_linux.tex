
%使用xelatex编译
%版权所有,翻版必究
%本文件由程序自动生成,任何修改将被覆盖
%2019 年 01 月 09 日




%

\FloatBarrier
\subsection{
在Linux平台下搭建开发环境
}\label{s000210}


Linux有众多发行版,
如果读者是首次在Linux平台下搭建Qt开发环境,建
议读者使用Ubuntu等使用者较多的版本。

\FloatBarrier
\subsubsection{
在Linux平台下安装Qt
}\label{ss000410}



如\commandnumbernameone\ \ref{command000000},所示:

\begin{itemize}
\item 第1行命令用于安装基本C{\sourcefonttwo{}+}{\sourcefonttwo{}+}开发环境;
\item 第2行命令用于安装OpenGL环境;
\end{itemize}

%\begin{spacing}{1.0}
%\FloatBarrier
\refstepcounter{commandnumber}\label{command000000}    %增加命令行编号
\begin{lstlisting}[escapeinside={(*@}{@*)},
caption=GoodLuck,
title=\commandnumbernameone\ \thecommandnumber
]
sudo apt-get install build-essential
sudo apt-get install libgl1-mesa-dev(*@\marginpar[\hfill\fbox{\footnotesize{\commandnumbernameone\ \thecommandnumber}}]{\fbox{\footnotesize{\commandnumbernameone\ \thecommandnumber}}}@*)\end{lstlisting}          %抄录环境
%\end{spacing}


读者可以参照\ \ref{ss000110}
节相关内容下载最新的Qt开发包并安装,如\commandnumbernameone\ \ref{command000001}所示:

\begin{itemize}
\item 第1行命令用于赋予Qt安装包执行权限;
\item 第2行运行安装包;
\end{itemize}

%\begin{spacing}{1.0}
%\FloatBarrier
\refstepcounter{commandnumber}\label{command000001}    %增加命令行编号
\begin{lstlisting}[escapeinside={(*@}{@*)},
caption=GoodLuck,
title=\commandnumbernameone\ \thecommandnumber
]
chmod +x qt-opensource-linux-x64-5.12.0.run
./qt-opensource-linux-x64-5.12.0.run(*@\marginpar[\hfill\fbox{\footnotesize{\commandnumbernameone\ \thecommandnumber}}]{\fbox{\footnotesize{\commandnumbernameone\ \thecommandnumber}}}@*)\end{lstlisting}          %抄录环境
%\end{spacing}


在不同Linux发行版本这些命令有所不同,
即使是同一发行版本,
随着时间推移命令也会有所变化。

读者可以访问\ \url{https://wiki.qt.io/Main}
获得更加详细的帮助。




\FloatBarrier
\subsubsection{
在Linux平台下安装Boost
}\label{ss000510}



在Linux下安装Boost极其简单,只需要执
行\commandnumbernameone\ \ref{command000002}即可。

%\begin{spacing}{1.0}
%\FloatBarrier
\refstepcounter{commandnumber}\label{command000002}    %增加命令行编号
\begin{lstlisting}[escapeinside={(*@}{@*)},
caption=GoodLuck,
title=\commandnumbernameone\ \thecommandnumber
]
sudo apt-get install libboost-all-dev(*@\marginpar[\hfill\fbox{\footnotesize{\commandnumbernameone\ \thecommandnumber}}]{\fbox{\footnotesize{\commandnumbernameone\ \thecommandnumber}}}@*)\end{lstlisting}          %抄录环境
%\end{spacing}



%不过相对于Windows平台 ,
%Linux这种直接使用全局库
%反而比Windows下直接拷贝
%隐含了更多问题。










%使用xelatex编译
%版权所有,翻版必究
%本文件由程序自动生成,任何修改将被覆盖
%2019 年 01 月 09 日



