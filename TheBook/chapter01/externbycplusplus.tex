
%使用XeLaTeX编译
%版权所有,翻版必究
%本文件由程序自动生成,任何修改将被覆盖
%2019 年 01 月 23 日




\FloatBarrier
\section{
从C{\sourcefonttwo{}+}{\sourcefonttwo{}+}导出数据
}\label{s100810t01}


%\begin{spacing}{1.0}
\refstepcounter{filesourcenumber}\label{f000081}    %增加源代码编号
\FloatBarrier                                  %强制完成浮动体布局
\begin{thebookfilesourceone}[escapeinside={(*@}{@*)},
caption=GoodLuck,
title=\filesourcenumbernameone \thefilesourcenumber
]
/*externbycplusplus/main.qml*/
import QtQuick 2.9
import sstd.quick 1.0

Rectangle{

    width: 512 ;
    height: 512 ;
    color: Qt.rgba(1,0,0,1);

    QuickMemoryImage{
        id : idQuickMemoryImage ;
        objectName: "quick_memory_image";
    }

    Image {
        anchors.centerIn: parent ;
        width: parent.width * 0.8 ;
        height: parent.height * 0.8 ;
        source: idQuickMemoryImage.imageName;
        fillMode: Image.PreserveAspectFit;
    }

    Image {
        anchors.centerIn: parent ;
        width: parent.width * 0.8 ;
        height: parent.height * 0.8 ;
        source: contex_quick_memory_image.imageName;
        fillMode: Image.PreserveAspectFit;
    }

}/*Rectangle*/(*@\marginpar[\hfill\setlength\fboxsep{2pt}\fbox{\footnotesize{\kaishu\parbox{1em}{\setlength{\baselineskip}{2pt}\filesourcenumbernameone}}\footnotesize{\thefilesourcenumber}}]{\setlength\fboxsep{2pt}\fbox{\footnotesize{\kaishu\parbox{1em}{\setlength{\baselineskip}{2pt}\filesourcenumbernameone}}\footnotesize{\thefilesourcenumber}}}@*)\end{thebookfilesourceone}          %抄录环境
\addtocounter{lstlisting}{-1}   %sub lstlisting counter ...
%\end{spacing}


%\begin{spacing}{1.0}
\refstepcounter{filesourcenumber}\label{f000082}    %增加源代码编号
\FloatBarrier                                  %强制完成浮动体布局
\begin{thebookfilesourceone}[escapeinside={(*@}{@*)},
caption=GoodLuck,
title=\filesourcenumbernameone \thefilesourcenumber
]
#include <sstd_qt_and_qml_library.hpp>

int main(int argc, char ** argv) {

    /*初始化程序*/
    auto varApp = sstd_make_unique< sstd::Application >(argc, argv);
    /*初始化Qml/Quick引擎*/
    auto varWindow = sstd_make_unique< sstd::DefaultRoowWindow >();
    {
        /*获得Qml文件绝对路径*/
        auto varFullFileName = sstd::getLocalFileFullPath(
            QStringLiteral("myqml/externbycplusplus/main.qml"));
        {
            /*装载图像加载器*/
            auto varImageProvider = sstd::QuickMemoryImage::getQuickImageProvider();
            varWindow->getEngine()->addImageProvider(
                varImageProvider.first, varImageProvider.second);
        }
        {
            auto varImageObject = sstd_new<sstd::QuickMemoryImage>();
            /*设置测试图片*/
            QImage varTestImage{ 512 , 256 ,QImage::Format_RGBA64_Premultiplied };
            varTestImage.fill(QColor(QRgba64::fromRgba64(1, 5535, 32215, 65535)));
            varImageObject->setImage(varTestImage);
            /*从C++端注册对象*/
            varWindow->getRootContext()
                ->setContextProperty(QStringLiteral("contex_quick_memory_image"), varImageObject);
        }
        /*加载Qml文件*/
        varWindow->load(varFullFileName);
        /*检查并报错*/
        if (varWindow->status() != sstd::LoadState::Ready) {
            qWarning() << QStringLiteral("can not load : ") << varFullFileName;
            return -1;
        }
        {
            /*从Qml端获得对象*/
            auto varRootItem = varWindow->getRootObject();
            auto varImageObject = varRootItem->findChild<sstd::QuickMemoryImage*>(
                QStringLiteral("quick_memory_image"));
            assert(varImageObject);
            /*设置测试图片*/
            QImage varTestImage{ 256 , 512 ,QImage::Format_RGBA64_Premultiplied };
            varTestImage.fill(QColor(QRgba64::fromRgba64(1, 32215, 5535, 65535)));
            varImageObject->setImage(varTestImage);
        }
    }
    varWindow->show();

    return varApp->exec();

}(*@\marginpar[\hfill\setlength\fboxsep{2pt}\fbox{\footnotesize{\kaishu\parbox{1em}{\setlength{\baselineskip}{2pt}\filesourcenumbernameone}}\footnotesize{\thefilesourcenumber}}]{\setlength\fboxsep{2pt}\fbox{\footnotesize{\kaishu\parbox{1em}{\setlength{\baselineskip}{2pt}\filesourcenumbernameone}}\footnotesize{\thefilesourcenumber}}}@*)\end{thebookfilesourceone}          %抄录环境
\addtocounter{lstlisting}{-1}   %sub lstlisting counter ...
%\end{spacing}


%begin图片
\begin{figure}[htb] %浮动体 here and top ...
%there must use marginnote ...
\marginnote{\setlength\fboxsep{2pt}\fbox{\footnotesize{\kaishu\figurename\,}\footnotesize{\ref{p000064}}}}\centering %中心对齐
\setlength\fboxsep{0pt}\fcolorbox[rgb]{0,0,0}{0.97,0.98,0.99}{\includegraphics[width=0.95\textwidth]{the_book_image/p000064.pdf}} %图片路径
\caption{bigscene} %标题
\label{p000064} %索引
\end{figure}
%end图片



% ______all_key_words
% the_book_chapter the_book_subsection the_book_subsubsection
% the_book_section the_book_image the_book_table
% the_book_file the_book_tree_file the_book_command_file
% littlelongworld tabbing ref
% figurename tablename filesourcenumbernameone
% treeindexnumbernameone commandnumbernameone footnote
% item itemize comment textbullet
% \hspace*{\parindent}
% FloatBarrier







%使用XeLaTeX编译
%版权所有,翻版必究
%本文件由程序自动生成,任何修改将被覆盖
%2019 年 01 月 23 日



