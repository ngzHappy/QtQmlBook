
%使用xelatex编译
%版权所有,翻版必究
%本文件由程序自动生成,任何修改将被覆盖
%2019 年 01 月 09 日




%

\FloatBarrier
\subsection{
第一个程序
}\label{s100210}



读者可以使用Qt Creator打开“QtQmlBook.pro”。
在Windows平台下,读者
也可以修改“build\underline{\hspace{0.5em}}msvc.bat”,从而使用Visual Studio。

如\lstlistingname\ \ref{f000018} :


%\begin{spacing}{1.0}
\begin{lstlisting}[label=f000018,
caption=GoodLuck,
title=\lstlistingname\ \thelstlisting
]
call "C:/Qt/Qt5.12.0/5.12.0/msvc2017_64/bin/qtenv2.bat"
cd /D "E:/QtQmlBookMsvc"
qmake -r -tp vc "E:/QtQmlBook/QtQmlBook.pro"
qmake -r -tp vc "E:/QtQmlBook/QtQmlMultimedia.pro"
qmake -r -tp vc "E:/QtQmlBook/QtQmlBookTest.pro"
qmake -r -tp vc "E:/QtQmlBook/TheBook/TheBook.pro"
cmd
\end{lstlisting}          %抄录环境
%\end{spacing}



\begin{itemize}

\item 第1行用于设置Qt运行环境;
\item 第2行用于设置Visual Studio工程文件输出目录;
\item 第3\raisebox{0.16ex}{\sourcefonttwo\~{}}6行用于指明将哪些qmake项目转为Visual Studio项目;

\end{itemize}






































%使用xelatex编译
%版权所有,翻版必究
%本文件由程序自动生成,任何修改将被覆盖
%2019 年 01 月 09 日



