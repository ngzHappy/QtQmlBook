
%使用xelatex编译
%版权所有,翻版必究
%本文件由程序自动生成,任何修改将被覆盖
%2019 年 01 月 09 日




\FloatBarrier
\section{
在Qt Quick中使用着色器
}\label{s100610}




\begin{figure}[htb] %浮动体 here and top ...
\centering %中心对齐
\includegraphics[width=0.95\textwidth]{../chapter01/shadereffect/the_app.png} %图片路径
\caption{Qt Quick中使用着色器} %标题
\label{p000008} %索引
\end{figure}


%\begin{spacing}{1.0}
\FloatBarrier
\begin{lstlisting}[label=f000038,
caption=GoodLuck,
title=\lstlistingname\ \thelstlisting
]
/*shadereffect/main.qml*/
import QtQuick 2.9

Rectangle {

    width: 640;
    height: 480;
    color: Qt.rgba(0.8,0.8,0.8,1);

    Rectangle{
        anchors.centerIn: parent    ;
        width: parent.width * 0.8   ;
        height: parent.height * 0.8 ;
        ShaderEffect{
            anchors.fill: parent ;
            fragmentShader:"
/*片段着色器*/
#version 460

in vec2  qt_TexCoord0/*纹理坐标*/  ;
out vec4 fragColor   /*输出值*/    ;

uniform float qt_Opacity/*透明度*/ ;

void main() {
    vec4 varColor  = vec4( qt_TexCoord0.x ,  qt_TexCoord0.y , 0.5 , 1);
    fragColor = varColor * qt_Opacity;
}

"
            vertexShader :"
/*顶点着色器*/
#version 460

in vec4 qt_Vertex/*输入点坐标*/    ;
out vec2 qt_TexCoord0/*纹理坐标*/  ;

uniform mat4 qt_Matrix/*投影矩阵*/ ;

void main() {
    gl_Position = qt_Matrix * qt_Vertex;
    qt_TexCoord0 = gl_Position.xy*0.5 + 0.5 ;
}

"
        }
    }

}/*~Rectangle*/
\end{lstlisting}          %抄录环境
%\end{spacing}
%main.qml


\begin{figure}[htb] %浮动体 here and top ...
\centering %中心对齐
\includegraphics[width=0.95\textwidth]{../chapter01/shadereffectwithimage/the_app.png} %图片路径
\caption{Qt Quick着色器中增加图片} %标题
\label{p000009} %索引
\end{figure}


%\begin{spacing}{1.0}
\FloatBarrier
\begin{lstlisting}[label=f000039,
caption=GoodLuck,
title=\lstlistingname\ \thelstlisting
]
/*shadereffectwithimage/main.qml*/
import QtQuick 2.9

Rectangle {

    width: 640;
    height: 480;
    color: Qt.rgba(0.8,0.8,0.8,1);

    Rectangle{
        anchors.centerIn: parent    ;
        width: parent.width * 0.8   ;
        height: parent.height * 0.8 ;
        Image{
            id : idSourceImage;
            source: "0000.jpg";
            visible: false    ;
        }
        ShaderEffect{
            property variant source: idSourceImage/*the image...*/
            anchors.fill: parent ;
            fragmentShader:"
/*片段着色器*/
#version 460

in vec2 qt_TexCoord0;

out vec4 fragColor;

uniform sampler2D source/*the image...*/;
uniform float qt_Opacity;

void main() {
    fragColor = texture(source, qt_TexCoord0) * qt_Opacity;
}

"
            vertexShader :"
/*顶点着色器*/
#version 460

in vec4 qt_Vertex;
in vec2 qt_MultiTexCoord0;

out vec2 qt_TexCoord0;

uniform mat4 qt_Matrix;

void main() {
    qt_TexCoord0 = qt_MultiTexCoord0;
    gl_Position = qt_Matrix * qt_Vertex;
}

"
        }
    }

}/*~Rectangle*/
\end{lstlisting}          %抄录环境
%\end{spacing}
%main.qml


























%使用xelatex编译
%版权所有,翻版必究
%本文件由程序自动生成,任何修改将被覆盖
%2019 年 01 月 09 日



