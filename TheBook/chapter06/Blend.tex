
%使用XeLaTeX编译
%版权所有,翻版必究
%本文件由程序自动生成,任何修改将被覆盖
%2019 年 01 月 23 日




\FloatBarrier
\section{
Blend
}\label{c000015s000002}


Blend特效常用属性
如\tablename\ \ref{tb000001}:


%使用XeLaTeX编译
%版权所有,翻版必究
%本文件由程序自动生成,任何修改将被覆盖
%2019 年 01 月 23 日



%begin表
\FloatBarrier                                  %强制完成浮动体布局
\begin{longtable}{cc}

%表头....
\toprule{}名称
&
描述[公式]%there must use marginnote ...
\marginnote{\setlength\fboxsep{2pt}\fbox{\footnotesize{\kaishu\tablename\,}\footnotesize{\ref{tb000002}}}}
\\ \midrule 
\endfirsthead

%表尾...
\bottomrule
\caption{混合类型}\label{tb000002} 
\endlastfoot

%重复表头
\toprule{}名称
&
描述[公式]
\\ \midrule
\endhead
%重复表尾
\midrule
\endfoot 
normal
    &
使用Alpha通道进行混合   
    \\

addition
    &
结果是source与foregroundSource的和
    \\

average
    &
结果是source与foregroundSource的均值
    \\

color
    &
结果是source的亮度值,
foregroundSource的色调与饱和度
    \\

colorBurn
    &
颜色加深$\left[v=1-(1-b)/a\right]$
 %与Photoshop colorBurn效果一致
    \\

colorDodge
    &
颜色减淡 $\left[v=b/(1-a)\right]$
 %与Photoshop colorDodge效果一致
    \\

darken
    &
变暗$\left[v=min(a,b)\right]$
 %与Photoshop  效果一致
    \\

darkerColor
    &
深色$\left[v=\begin{cases}
a, & a_r+a_r+a_b<b_r+b_g+b_b \\ 
b, & a_r+a_r+a_b>b_r+b_g+b_b
\end{cases}\right]$
 %与Photoshop  效果一致
    \\

difference
    &
    &
变暗$\left[v=abs(b-a)\right]$
 %与Photoshop  效果一致
    \\

divide
    &
    &
划分$\left[v=b/a\right]$
 %与Photoshop  效果一致
    \\

exclusion
    &
    &
排除$\left[v=b/a\right]$
 %与Photoshop  效果一致
    \\
\end{longtable}
%end表





%使用XeLaTeX编译
%版权所有,翻版必究
%本文件由程序自动生成,任何修改将被覆盖
%2019 年 01 月 23 日





混合类型
如\tablename\ \ref{tb000002}
    \footnote{$a$代表foregroundSource,
$b$代表source,
$0$代表黑色,
$1$代表白色,
$v$代表最终结果。}
:


%使用XeLaTeX编译
%版权所有,翻版必究
%本文件由程序自动生成,任何修改将被覆盖
%2019 年 01 月 23 日



%begin表
\FloatBarrier                                  %强制完成浮动体布局
\begin{longtable}{cc}

%表头....
\toprule{}名称
&
描述[公式]%there must use marginnote ...
\marginnote{\setlength\fboxsep{2pt}\fbox{\footnotesize{\kaishu\tablename\,}\footnotesize{\ref{tb000002}}}}
\\ \midrule 
\endfirsthead

%表尾...
\bottomrule
\caption{混合类型}\label{tb000002} 
\endlastfoot

%重复表头
\toprule{}名称
&
描述[公式]
\\ \midrule
\endhead
%重复表尾
\midrule
\endfoot 
normal
    &
使用Alpha通道进行混合   
    \\

addition
    &
结果是source与foregroundSource的和
    \\

average
    &
结果是source与foregroundSource的均值
    \\

color
    &
结果是source的亮度值,
foregroundSource的色调与饱和度
    \\

colorBurn
    &
颜色加深$\left[v=1-(1-b)/a\right]$
 %与Photoshop colorBurn效果一致
    \\

colorDodge
    &
颜色减淡 $\left[v=b/(1-a)\right]$
 %与Photoshop colorDodge效果一致
    \\

darken
    &
变暗$\left[v=min(a,b)\right]$
 %与Photoshop  效果一致
    \\

darkerColor
    &
深色$\left[v=\begin{cases}
a, & a_r+a_r+a_b<b_r+b_g+b_b \\ 
b, & a_r+a_r+a_b>b_r+b_g+b_b
\end{cases}\right]$
 %与Photoshop  效果一致
    \\

difference
    &
    &
变暗$\left[v=abs(b-a)\right]$
 %与Photoshop  效果一致
    \\

divide
    &
    &
划分$\left[v=b/a\right]$
 %与Photoshop  效果一致
    \\

exclusion
    &
    &
排除$\left[v=b/a\right]$
 %与Photoshop  效果一致
    \\
\end{longtable}
%end表





%使用XeLaTeX编译
%版权所有,翻版必究
%本文件由程序自动生成,任何修改将被覆盖
%2019 年 01 月 23 日





%begin图片
\begin{figure}[htb] %浮动体 here and top ...
%there must use marginnote ...
\marginnote{\setlength\fboxsep{2pt}\fbox{\footnotesize{\kaishu\figurename\,}\footnotesize{\ref{p000018}}}}\centering %中心对齐
\includegraphics[width=0.95\textwidth]{../chapter06/blend_effect/the_app.png} %图片路径
\caption{Blend} %标题
\label{p000018} %索引
\end{figure}
%end图片


%\begin{spacing}{1.0}
\refstepcounter{filesourcenumber}\label{f000052}    %增加源代码编号
\FloatBarrier                                  %强制完成浮动体布局
\begin{thebookfilesourceone}[escapeinside={(*@}{@*)},
caption=GoodLuck,
title=\filesourcenumbernameone \thefilesourcenumber
]
/*blend_effect/main.qml*/
import QtQuick 2.9
import QtGraphicalEffects 1.12

Rectangle {
    id : idRoot
    width: 640;
    height: 480;
    color: Qt.rgba(0.8,0.8,0.8,1);

    Image{
        anchors.fill: parent;
        source: "grass.jpg"
        fillMode: Image.Tile
        id : idGrass
        visible: false
    }

    Image{
        anchors.centerIn: parent;
        source: "bear.png"
        fillMode: Image.Stretch
        id : idBear
        visible: false
    }

    Blend{
        source: idGrass
        foregroundSource: idBear
        mode: idBlendControl.blendModeComboBox.currentText
        anchors.centerIn: parent;
        width: idBear.width
        height: idBear.height
    }

    BlendControl {
        id : idBlendControl
    }

}(*@\marginpar[\hfill\setlength\fboxsep{2pt}\fbox{\footnotesize{\kaishu\parbox{1em}{\setlength{\baselineskip}{2pt}\filesourcenumbernameone}}\footnotesize{\thefilesourcenumber}}]{\setlength\fboxsep{2pt}\fbox{\footnotesize{\kaishu\parbox{1em}{\setlength{\baselineskip}{2pt}\filesourcenumbernameone}}\footnotesize{\thefilesourcenumber}}}@*)\end{thebookfilesourceone}          %抄录环境
\addtocounter{lstlisting}{-1}   %sub lstlisting counter ...
%\end{spacing}


 






%使用XeLaTeX编译
%版权所有,翻版必究
%本文件由程序自动生成,任何修改将被覆盖
%2019 年 01 月 23 日



