
%使用XeLaTeX编译
%版权所有,翻版必究
%本文件由程序自动生成,任何修改将被覆盖
%2019 年 01 月 23 日



%begin表
\FloatBarrier                                  %强制完成浮动体布局
\begin{longtable}{cc}

%表头....
\toprule{}简介
&
描述%there must use marginnote ...
\marginnote{\setlength\fboxsep{2pt}\fbox{\footnotesize{\kaishu\tablename\,}\footnotesize{\ref{tb000002}}}}
\\ \midrule 
\endfirsthead

%表尾...
\bottomrule
\caption{混合类型}\label{tb000002} 
\endlastfoot

%重复表头
\toprule{}简介
&
描述
\\ \midrule
\endhead
%重复表尾
\midrule
\endfoot 
normal
    &
使用Alpha通道进行混合   
    \\

addition
    &
结果是source与foregroundSource的和
    \\


average
    &
结果是source与foregroundSource的均值
    \\

color
    &
结果是source的亮度值,
foregroundSource的色调与饱和度
    \\

colorBurn
    &
颜色加深  %与Photoshop colorBurn效果一致
    \\

colorDodge
    &
颜色减淡  %与Photoshop colorDodge效果一致
    \\
\end{longtable}
%end表





%使用XeLaTeX编译
%版权所有,翻版必究
%本文件由程序自动生成,任何修改将被覆盖
%2019 年 01 月 23 日



