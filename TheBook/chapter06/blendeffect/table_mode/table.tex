
%使用XeLaTeX编译
%版权所有,翻版必究
%本文件由程序自动生成,任何修改将被覆盖
%2019 年 01 月 23 日



%begin表
\FloatBarrier                                  %强制完成浮动体布局
\begin{longtable}{cc}

%表头....
\toprule{}模式
&
描述[公式]%there must use marginnote ...
\marginnote{\setlength\fboxsep{2pt}\fbox{\footnotesize{\kaishu\tablename\,}\footnotesize{\ref{tb000002}}}}
\\ \midrule 
\endfirsthead

%表尾...
\endlastfoot

%重复表头
\toprule{}模式
&
描述[公式]
\\ \midrule
\endhead
%重复表尾
\midrule
\endfoot 
normal
    &
结果是foregroundSource的rgb,
alpha取二者中alpha的较大者
    \\

addition
    &
结果是source与foregroundSource的和
    \\

average
    &
结果是source与foregroundSource的均值
    \\

saturation
    &
结果是source的亮度值与色调,
foregroundSource的饱和度
    \\

hue
    &
结果是source的亮度值与饱和度,
foregroundSource的色调
    \\

color
    &
结果是source的亮度值,
foregroundSource的色调与饱和度
    \\

lightness
    &
结果是foregroundSource的亮度值,
source的色调与饱和度
    \\

colorBurn
    &
颜色加深$\left[v=1-(1-b)/a\right]$
 %与Photoshop colorBurn效果一致
    \\

colorDodge
    &
颜色减淡 $\left[v=b/(1-a)\right]$
 %与Photoshop colorDodge效果一致
    \\

darken
    &
变暗$\left[v=min(a,b)\right]$
 %与Photoshop  效果一致
    \\

lighten
    &
变亮$\left[v=max(a,b)\right]$
 %与Photoshop  效果一致
    \\

darkerColor
    &
深色$\left[v=\begin{cases}
a, & a_r+a_r+a_b<b_r+b_g+b_b \\ 
b, & a_r+a_r+a_b>b_r+b_g+b_b
\end{cases}\right]$
 %与Photoshop  效果一致
    \\

lighterColor
    &
浅色$\left[v=\begin{cases}
a, & a_r+a_r+a_b>b_r+b_g+b_b \\ 
b, & a_r+a_r+a_b<b_r+b_g+b_b
\end{cases}\right]$
 %与Photoshop  效果一致
    \\

difference
    &
变暗$\left[v=\left|b-a\right|\right]$
 %与Photoshop  效果一致
    \\

divide
    &
划分$\left[v=b/a\right]$
 %与Photoshop  效果一致
    \\

multiply
    &
正片叠底$\left[v=b\times{}a\right]$
 %与Photoshop  效果一致
    \\

negation
    &
否定$\left[v=1/\left|b-a\right|\right]$
 %与Photoshop  效果一致
    \\

exclusion
    &
排除$\left[v=a+b-(a\times{}b)/2\right]$
 %与Photoshop  效果一致
    \\

hardLight
    &
强光$\left[v=\begin{cases}
2ab, & a<0.5 \\ 
1-2(1-a)(1-b), & a\ge{}0.5
\end{cases}\right]$
 %与Photoshop  效果一致
    \\

softLight
    &
柔光$\left[v=\begin{cases}
2ab+a^2(1-2b), & a<0.5 \\ 
2a(1-b)+\sqrt{a}(2a-1), & a\ge{}0.5
\end{cases}\right]$
 %与Photoshop  效果一致
    \\

screen
    &
滤色$\left[v=1-(1-a)(1-b)\right]$
 %与Photoshop  效果一致
    \\

subtract
    &
减去$\left[v=b-a\right]$
 %与Photoshop  效果一致
    \\
\bottomrule            %表底部线
\caption{类型属性}\label{tb000002} %表标题
\end{longtable}
%end表





%使用XeLaTeX编译
%版权所有,翻版必究
%本文件由程序自动生成,任何修改将被覆盖
%2019 年 01 月 23 日



